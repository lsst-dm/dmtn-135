\tiny \begin{longtable} { |p{0.22\textwidth}  |r  |r  |r  |r  |r |}
\caption{This table pulls together all the information in a high level summary - in this table Xeon pricing is used since that is the more expensive but better known option. Price factors, defined in \tabref{tab:Inputs} are applied post 2020.
 \label{tab:Summary}}\\
\hline
\textbf{Year}&\textbf{2020}&\textbf{2021}&\textbf{2022}&\textbf{2023} \\ \hline
{Compute (2019 pricing)}&{\$300,102}&{\$0}&{\$3,286,841}&{\$7,976,643} \\ \hline
{Applying price factor (CPU)}&{}&{\$0}&{\$2,629,473}&{\$5,583,650} \\ \hline
{IN2P3 (50\% of compute)}&{}&{}&{}&{-\$2,791,825} \\ \hline
{Storage (2019 pricing)}&{\$164,213}&{\$245,799}&{\$1,760,037}&{\$9,281,372} \\ \hline
{Applying price factor (Storage)}&{\$164,213}&{\$233,509}&{\$1,584,034}&{\$7,889,166} \\ \hline
\textbf{Total budget (using price factors)}&\textbf{\$164,213}&\textbf{\$233,509}&\textbf{\$4,213,506}&\textbf{\$10,680,991} \\ \hline
\end{longtable} \normalsize
