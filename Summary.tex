\tiny \begin{longtable} { |p{0.22\textwidth}  |r  |r  |r  |r  |r |} 
\caption{This table pulls together all the information in a high level summary - in this table Xeon pricing is used since that is the more expensive but better known option. Price factors, defined in \tabref{tab:Inputs} are applied post 2020.
 \label{tab:Summary}}\\ 
\hline 
\textbf{Year}&\textbf{2020}&\textbf{2021}&\textbf{2022}&\textbf{2023} \\ \hline
{Compute (2019 pricing)}&{\$690,000}&{\$0}&{\$1,286,151}&{\$2,826,893} \\ \hline
{Applying price factor (CPU)}&{}&{\$0}&{\$1,028,921}&{\$1,978,825} \\ \hline
{IN2P3 (50\% of compute)}&{}&{}&{}&{-\$989,412} \\ \hline
{Qserv (2019 pricing)}&{}&{}&{\$560,000}&{\$3,240,000} \\ \hline
{Qserv (applying factor)}&{}&{}&{\$476,000}&{\$2,511,000} \\ \hline
{Storage (2019 pricing)}&{\$333,952}&{\$333,863}&{\$1,813,698}&{\$11,316,420} \\ \hline
{Applying price factor (Storage)}&{\$333,952}&{\$317,170}&{\$1,632,328}&{\$9,618,957} \\ \hline
{Hosting cost NCSA
}&{\$119,855}&{\$71,855}&{\$237,447}&{\$537,367} \\ \hline
\textbf{Total budget (using price factors)}&\textbf{\$453,807}&\textbf{\$389,025}&\textbf{\$3,209,104}&\textbf{\$13,656,736} \\ \hline
\end{longtable} \normalsize
