\tiny \begin{longtable} { |p{0.22\textwidth}  |r  |r  |r  |r |} 
\caption{This table pulls together all the information in a high level summary - in this table Xeon pricing is used since that is the more expensive but better known option. Price factors, defined in \tabref{tab:Inputs} are applied post 2020.
 \label{tab:Summary}}\\ 
\hline 
\textbf{Year}&\textbf{2021}&\textbf{2022}&\textbf{2023} \\ \hline
{Compute (2019 pricing)}&{\$690,000}&{\$0}&{\$1,500,000} \\ \hline
{Storage (2019 pricing)}&{\$183,054}&{\$138,290}&{\$1,450,989} \\ \hline
{Qserv (2019 pricing)}&{}&{}&{\$280,000} \\ \hline
\textbf{Total (2019 pricing)}&\textbf{\$873,054}&\textbf{\$138,290}&\textbf{\$3,230,989} \\ \hline
{Compute (applying price factor)}&{\$552,000}&{\$0}&{\$900,000} \\ \hline
{IN2P3 (50\% of compute in ops)}&{}&{}&{} \\ \hline
{UKDF (25\% of compute in ops)}&{}&{}&{} \\ \hline
{Storage (applying price factor)}&{\$155,596}&{\$107,175}&{\$1,015,692} \\ \hline
{Qserv (applying price factor)}&{}&{}&{\$182,000} \\ \hline
{Hosting cost NCSA
}&{\$110,802}&{\$62,802}&{\$238,012} \\ \hline
\textbf{Total budget (using price factors)}&\textbf{\$818,398}&\textbf{\$169,977}&\textbf{\$2,335,705} \\ \hline
\end{longtable} \normalsize
