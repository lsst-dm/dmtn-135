\tiny \begin{longtable} { |p{0.22\textwidth}  |r  |r  |r  |r  |r |} 
\caption{This table pulls together all the information in a high level summary - in this table Xeon pricing is used since that is the more expensive but better known option. Price factors, defined in \tabref{tab:Inputs} are applied post 2020.
 \label{tab:Summary}}\\ 
\hline 
\textbf{Year}&\textbf{2020}&\textbf{2021}&\textbf{2022}&\textbf{2023} \\ \hline
{Compute (2019 pricing)}&{\$683,079}&{\$0}&{\$3,089,169}&{\$9,376,796} \\ \hline
{Applying price factor (CPU)}&{}&{\$0}&{\$2,471,335}&{\$6,563,757} \\ \hline
{IN2P3 (50\% of compute)}&{}&{}&{}&{-\$3,281,879} \\ \hline
{Storage (2019 pricing)}&{\$450,901}&{\$366,113}&{\$1,912,599}&{\$12,818,980} \\ \hline
{Applying price factor (Storage)}&{\$450,901}&{\$347,807}&{\$1,721,339}&{\$10,896,133} \\ \hline
\textbf{Total budget (using price factors)}&\textbf{\$450,901}&\textbf{\$347,807}&\textbf{\$4,192,674}&\textbf{\$14,178,012} \\ \hline
\end{longtable} \normalsize
