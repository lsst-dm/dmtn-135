\subsection{Ops Cost details}\label{sec:opsdetails}
\tabref{tab:opsXeon} gives the price of compute based on Xeons.
This is broken down further for US in \tabref{tab:opsXeonUSDF} and Chile in \tabref{tab:opsXeonChile}.
However the Ops costing for SLAC was done using \tabref{tab:opsRomeUSDF}.

\tabref{tab:opsStorageCost} gives the price of storage using all  types that we need.
This is broken down further for US in \tabref{tab:opsStorageUSDF} and Chile in \tabref{tab:opsStorageChile}
This would be needed regardless of the compute chosen.
\tiny \begin{longtable} { |p{0.22\textwidth}  |r  |r  |r  |r  |r  |r  |r  |r  |r  |r  |r |} 
\caption{Implementation with Intel Xeon \label{tab:opsXeon}}\\ 
\hline 
\textbf{Year}&\textbf{2023}&\textbf{2024}&\textbf{2025}&\textbf{2026}&\textbf{2027}&\textbf{2028}&\textbf{2029}&\textbf{2030}&\textbf{2031}&\textbf{2032} \\ \hline
{Number of Xeon}&{283}&{300}&{472}&{621}&{637}&{698}&{675}&{675}&{698}&{675} \\ \hline
{Approximate cost (2019 Mdollars)}&{\$2.83}&{\$3.00}&{\$4.72}&{\$6.21}&{\$6.37}&{\$6.98}&{\$6.75}&{\$6.75}&{\$6.98}&{\$6.75} \\ \hline
\end{longtable} \normalsize

\tiny \begin{longtable} { |p{0.22\textwidth}  |r  |r  |r  |r  |r  |r  |r  |r  |r  |r  |r |} 
\caption{Implementation with Rome for USDF \label{tab:opsRomeUSDF}}\\ 
\hline 
\textbf{Year}&\textbf{2024}&\textbf{2025}&\textbf{2026}&\textbf{2027}&\textbf{2028}&\textbf{2029}&\textbf{2030}&\textbf{2031}&\textbf{2032}&\textbf{2033} \\ \hline
{Number of Large Rome USDF DRP}&{62}&{66}&{98}&{135}&{139}&{147}&{147}&{147}&{147}&{147} \\ \hline
{Approximate cost (2020 Mdollars) DRP}&{\$0.62}&{\$0.66}&{\$0.98}&{\$1.35}&{\$1.39}&{\$1.47}&{\$1.47}&{\$1.47}&{\$1.47}&{\$1.47} \\ \hline
{Number of Large Rome USDF DAC}&{4}&{4}&{6}&{8}&{8}&{9}&{9}&{9}&{9}&{9} \\ \hline
{Approximate cost (2020 Mdollars) DAC}&{\$0.04}&{\$0.04}&{\$0.06}&{\$0.08}&{\$0.08}&{\$0.09}&{\$0.09}&{\$0.09}&{\$0.09}&{\$0.09} \\ \hline
{Number of Large Rome USDF AP}&{0}&{0}&{11}&{0}&{0}&{11}&{0}&{0}&{11}&{0} \\ \hline
{Approximate cost (2020 Mdollars) AP}&{\$0.00}&{\$0.00}&{\$0.11}&{\$0.00}&{\$0.00}&{\$0.11}&{\$0.00}&{\$0.00}&{\$0.11}&{\$0.00} \\ \hline
\end{longtable} \normalsize

\input{opsXeonUSDF}
\tiny \begin{longtable} { |p{0.22\textwidth}  |r  |r  |r  |r  |r  |r  |r  |r  |r  |r  |r |} 
\caption{Implementation with Intel Xeon for Chile Compute \label{tab:opsXeonChile}}\\ 
\hline 
\textbf{Year}&\textbf{2024}&\textbf{2025}&\textbf{2026}&\textbf{2027}&\textbf{2028}&\textbf{2029}&\textbf{2030}&\textbf{2031}&\textbf{2032}&\textbf{2033} \\ \hline
{Number of Xeon Chile}&{4}&{4}&{4}&{8}&{7}&{7}&{7}&{7}&{7}&{7} \\ \hline
{Approximate cost (2020 Mdollars)}&{\$0.04}&{\$0.04}&{\$0.04}&{\$0.08}&{\$0.07}&{\$0.07}&{\$0.07}&{\$0.07}&{\$0.07}&{\$0.07} \\ \hline
\end{longtable} \normalsize

\tiny \begin{longtable} { |p{0.22\textwidth}  |r  |r  |r  |r  |r  |r  |r  |r  |r  |r  |r |} 
\caption{Total storage cost estimate for operations of Rubin Observatory USDF and CHile \label{tab:opsStorageCost}}\\ 
\hline 
\textbf{Year (all in M\$)}&\textbf{2023}&\textbf{2024}&\textbf{2025}&\textbf{2026}&\textbf{2027}&\textbf{2028}&\textbf{2029}&\textbf{2030}&\textbf{2031}&\textbf{2032} \\ \hline
{Fast Storage}&{\$0.13}&{\$0.14}&{\$0.17}&{\$0.06}&{\$0.06}&{\$0.18}&{\$0.19}&{\$0.22}&{\$0.11}&{\$0.10} \\ \hline
{Normal Storage}&{\$3.96}&{\$3.86}&{\$4.20}&{\$4.19}&{\$4.89}&{\$8.13}&{\$8.05}&{\$8.41}&{\$8.42}&{\$8.41} \\ \hline
{Latent Storage}&{\$7.01}&{\$9.37}&{\$10.75}&{\$9.44}&{\$9.98}&{\$16.45}&{\$18.81}&{\$20.19}&{\$18.87}&{\$18.87} \\ \hline
{High Latency Storage}&{\$0.73}&{\$1.12}&{\$1.56}&{\$2.00}&{\$2.43}&{\$2.85}&{\$3.28}&{\$3.72}&{\$4.15}&{\$4.58} \\ \hline
\textbf{Total (M\$)}&\textbf{\$11.82}&\textbf{\$14.49}&\textbf{\$16.68}&\textbf{\$15.69}&\textbf{\$17.36}&\textbf{\$27.62}&\textbf{\$30.34}&\textbf{\$32.53}&\textbf{\$31.55}&\textbf{\$31.96} \\ \hline
\end{longtable} \normalsize

\tiny \begin{longtable} { |p{0.22\textwidth}  |r  |r  |r  |r  |r  |r  |r  |r  |r  |r  |r |} 
\caption{Total storage cost estimate for operations at USDF \label{tab:opsStorageUSDF}}\\ 
\hline 
\textbf{Year (all in M\$)}&\textbf{2024}&\textbf{2025}&\textbf{2026}&\textbf{2027}&\textbf{2028}&\textbf{2029}&\textbf{2030}&\textbf{2031}&\textbf{2032}&\textbf{2033} \\ \hline
{Fast Storage USDF}&{\$0.06}&{\$0.07}&{\$0.09}&{\$0.03}&{\$0.04}&{\$0.09}&{\$0.09}&{\$0.11}&{\$0.06}&{\$0.05} \\ \hline
{Normal Storage USDF}&{\$3.96}&{\$3.86}&{\$4.20}&{\$4.19}&{\$4.89}&{\$8.13}&{\$8.05}&{\$8.41}&{\$8.42}&{\$8.41} \\ \hline
{Latent Storage USDF}&{\$3.18}&{\$4.69}&{\$5.37}&{\$4.72}&{\$5.26}&{\$7.90}&{\$9.40}&{\$10.09}&{\$9.44}&{\$9.44} \\ \hline
{High Latency Storage USDF}&{\$0.73}&{\$1.13}&{\$1.56}&{\$2.00}&{\$2.43}&{\$2.86}&{\$3.28}&{\$3.72}&{\$4.15}&{\$4.58} \\ \hline
\textbf{Total (M\$)}&\textbf{\$7.92}&\textbf{\$9.73}&\textbf{\$11.22}&\textbf{\$10.94}&\textbf{\$12.62}&\textbf{\$18.97}&\textbf{\$20.83}&\textbf{\$22.33}&\textbf{\$22.05}&\textbf{\$22.47} \\ \hline
\end{longtable} \normalsize

\tiny \begin{longtable} { |p{0.22\textwidth}  |r  |r  |r  |r  |r  |r  |r  |r  |r  |r  |r |} 
\caption{Total storage cost estimate for operations in Chile  {\bf Note} the latent storage here is 1.05 of the Raw and LSSTCam Coadd image volume. \label{tab:opsStorageChile}}\\ 
\hline 
\textbf{Year (all in M\$)}&\textbf{2024}&\textbf{2025}&\textbf{2026}&\textbf{2027}&\textbf{2028}&\textbf{2029}&\textbf{2030}&\textbf{2031}&\textbf{2032}&\textbf{2033} \\ \hline
{Fast Storage Chile}&{\$0.11}&{\$0.07}&{\$0.07}&{\$0.07}&{\$0.06}&{\$0.16}&{\$0.12}&{\$0.13}&{\$0.13}&{\$0.11} \\ \hline
{Latent Storage Chile}&{\$0.74}&{\$0.55}&{\$0.53}&{\$0.35}&{\$0.35}&{\$1.09}&{\$0.91}&{\$0.88}&{\$0.71}&{\$0.71} \\ \hline
\textbf{Total (M\$)}&\textbf{\$0.84}&\textbf{\$0.62}&\textbf{\$0.60}&\textbf{\$0.42}&\textbf{\$0.41}&\textbf{\$1.25}&\textbf{\$1.03}&\textbf{\$1.01}&\textbf{\$0.83}&\textbf{\$0.82} \\ \hline
\end{longtable} \normalsize


\tabref{tab:opsOverheadCost} gives the annual cost of hosting compute in NCSA. This includes purchasing racks to house
new nodes etc.
\tiny \begin{longtable} { |p{0.22\textwidth}  |r  |r  |r  |r  |r |} 
\caption{Overheads(NCSA) per year based on number of cores in \tabref{tab:opsInputs} and costs in \tabref{tab:overheads} assuming Xeon density from \tabref{tab:Machines}.  \label{tab:opsOverheadCost}}\\ 
\hline 
\textbf{Year}&\textbf{2023}&\textbf{2024}&\textbf{2025}&\textbf{2026} \\ \hline
\textbf{Total Incremental cores (USA)}&\textbf{7,521}&\textbf{7,958}&\textbf{12,551}&\textbf{16,499} \\ \hline
\textbf{Total owned cores (USA)}&\textbf{14,478}&\textbf{22,437}&\textbf{31,415}&\textbf{40,394} \\ \hline
\textbf{Total owned nodes}&\textbf{548}&\textbf{918}&\textbf{1,291}&\textbf{1,611} \\ \hline
{Cost for hosting nodes}&{\$310,051}&{\$519,392}&{\$730,430}&{\$911,482} \\ \hline
\textbf{Total new nodes}&\textbf{317}&\textbf{370}&\textbf{486}&\textbf{636} \\ \hline
\textbf{Total new racks}&\textbf{9}&\textbf{11}&\textbf{14}&\textbf{18} \\ \hline
{Rack install cost }&{\$216,000.00}&{\$264,000.00}&{\$336,000.00}&{\$432,000.00} \\ \hline
\textbf{Total Ops Overhead (NCSA)}&\textbf{\$526,050.86}&\textbf{\$783,391.76}&\textbf{\$1,066,430.03}&\textbf{\$1,343,481.63} \\ \hline
\end{longtable} \normalsize


For Chile \tabref{tab:opsOverheadChile} gives the cost of hosting in Chile (las Serena).
\tiny \begin{longtable} { |p{0.22\textwidth}  |r  |r  |r  |r  |r  |r  |r  |r  |r  |r  |r |} 
\caption{Overheads(Chile) per year based on number of cores in \tabref{tab:opsInputs} and costs in \tabref{tab:overheads} assuming Xeon density from \tabref{tab:Machines}.  \label{tab:opsOverheadChile}}\\ 
\hline 
\textbf{Year}&\textbf{2024}&\textbf{2025}&\textbf{2026}&\textbf{2027}&\textbf{2028}&\textbf{2029}&\textbf{2030}&\textbf{2031}&\textbf{2032}&\textbf{2033} \\ \hline
\textbf{Total Incremental cores (Chile)}&\textbf{103}&\textbf{83}&\textbf{93}&\textbf{93}&\textbf{93}&\textbf{93}&\textbf{93}&\textbf{93}&\textbf{93}&\textbf{93} \\ \hline
\textbf{Total owned cores (Chile)}&\textbf{103}&\textbf{187}&\textbf{280}&\textbf{373}&\textbf{466}&\textbf{560}&\textbf{653}&\textbf{746}&\textbf{840}&\textbf{933} \\ \hline
{Compute nodes}&{4}&{6}&{9}&{12}&{15}&{18}&{21}&{24}&{27}&{30} \\ \hline
{Qserv nodes}&{95}&{216}&{309}&{348}&{364}&{451}&{436}&{408}&{367}&{418} \\ \hline
\textbf{Total Nodes}&\textbf{99}&\textbf{222}&\textbf{318}&\textbf{360}&\textbf{379}&\textbf{469}&\textbf{457}&\textbf{432}&\textbf{394}&\textbf{448} \\ \hline
\textbf{Total Compute Racks}&\textbf{3}&\textbf{7}&\textbf{9}&\textbf{10}&\textbf{11}&\textbf{14}&\textbf{13}&\textbf{12}&\textbf{11}&\textbf{13} \\ \hline
\textbf{Total Storage}&\textbf{8,751}&\textbf{10,786}&\textbf{12,430}&\textbf{10,992}&\textbf{11,007}&\textbf{19,751}&\textbf{21,649}&\textbf{23,172}&\textbf{21,655}&\textbf{21,704} \\ \hline
{Toital Storage Racks}&{2}&{2}&{2}&{2}&{2}&{3}&{3}&{3}&{3}&{3} \\ \hline
{Power kW
}&{60.48}&{108.864}&{133.056}&{145.152}&{157.248}&{205.632}&{193.536}&{181.44}&{169.344}&{193.536} \\ \hline
{Power Cost (CLP)}&{\$6,667}&{\$12,000}&{\$14,667}&{\$17,600}&{\$19,067}&{\$24,934}&{\$25,814}&{\$24,200}&{\$22,587}&{\$28,395} \\ \hline
{New Compute Rack Cost}&{\$85,902.00}&{\$114,536.00}&{\$57,268.00}&{\$28,634.00}&{\$28,634.00}&{\$85,902.00}&{\$0.00}&{\$0.00}&{\$0.00}&{\$0.00} \\ \hline
{New Storage Rack Cost}&{\$57,268.00}&{\$0.00}&{\$0.00}&{\$0.00}&{\$0.00}&{\$28,634.00}&{\$0.00}&{\$0.00}&{\$0.00}&{\$0.00} \\ \hline
\textbf{Total Ops Overhead Chile (USD)}&\textbf{\$149,837}&\textbf{\$126,536}&\textbf{\$71,935}&\textbf{\$46,234}&\textbf{\$47,701}&\textbf{\$139,470}&\textbf{\$25,814}&\textbf{\$24,200}&\textbf{\$22,587}&\textbf{\$28,395} \\ \hline
\end{longtable} \normalsize


For Chile the rack costs are outlined in \tabref{tab:rackCostChile}.
\tiny \begin{longtable} {{ | l | r| r |r|r| l|}} \caption{This table details the cost per rack which is added in \tabref{tab:opsOverheadChile}. \label{tab:rackCostChile}}\\ 
\hline 
\textbf{2020 Rack Component }&\textbf{Unit Cost}&\textbf{Spine Port}&\textbf{}&\textbf{Total}& \\ \hline
{2 x Leaf}&{\$7,000.00}&{\$2,187.00}&{\$9,187.00}&{\$18,374.00}&{Cisco Nexus 93108TC-EX, + overhead of Spine port} \\ \hline
{2 x PDU}&{\$2,980.00}&{}&{}&{\$5,960.00}&{Raritan PX3 5085U-N2} \\ \hline
{1 x IPMI}&{\$1,800.00}&{}&{}&{\$1,800.00}&{Cisco Catalyst 2960-X / Cisco 9200} \\ \hline
{1 x Rack}&{\$2,500.00}&{}&{}&{\$2,500.00}&{APC AR3357} \\ \hline
\textbf{Total}&\textbf{}&\textbf{}&\textbf{}&\textbf{\$28,634.00}& \\ \hline
\end{longtable} \normalsize


Various other inputs to ops costing are given in \tabref{tab:opsInputs}.
\tiny \begin{longtable} { |p{0.22\textwidth}  |r  |r  |r  |r  |r  |r  |r  |r  |r  |r  |r |} 
\caption{Various inputs for deriving costs in operations - these drive the costs in \tabref{tab:opsSummary}. This is based on \tabref{tab:opsXeon}, \tabref{tab:opsStorageCost}  \label{tab:opsInputs}}\\ 
\hline 
\textbf{Year}&\textbf{2024}&\textbf{2025}&\textbf{2026}&\textbf{2027}&\textbf{2028}&\textbf{2029}&\textbf{2030}&\textbf{2031}&\textbf{2032}&\textbf{2033} \\ \hline
{Core-hours Needed Total (DRP)}&{4.5E+07}&{8.2E+07}&{1.2E+08}&{1.6E+08}&{2.0E+08}&{2.5E+08}&{2.9E+08}&{3.3E+08}&{3.7E+08}&{4.1E+08} \\ \hline
{Core-hours Annual Increase}&{3.40E+07}&{3.6E+07}&{4.1E+07}&{4.1E+07}&{4.1E+07}&{4.1E+07}&{4.1E+07}&{4.1E+07}&{4.1E+07}&{4.1E+07} \\ \hline
{Time to Process days}&{200}&{200}&{200}&{200}&{200}&{200}&{200}&{200}&{200}&{200} \\ \hline
{Time to Process hours}&{4,800}&{4,800}&{4,800}&{4,800}&{4,800}&{4,800}&{4,800}&{4,800}&{4,800}&{4,800} \\ \hline
{Cores (DRP) Annual increase}&{7,093}&{7,594}&{8,512}&{8,512}&{8,512}&{8,512}&{8,512}&{8,512}&{8,512}&{8,512} \\ \hline
{Cores (DRP) Annual refresh}&{}&{}&{2,837}&{7,093}&{7,594}&{8,512}&{8,512}&{8,512}&{8,512}&{8,512} \\ \hline
{Cores (DRP) Annual purchase}&{7,093}&{7,594}&{11,349}&{15,605}&{16,106}&{17,024}&{17,024}&{17,024}&{17,024}&{17,024} \\ \hline
{Cores (Alerts)}&{1,188}&{1,188}&{1,188}&{1,188}&{1,188}&{1,188}&{1,188}&{1,188}&{1,188}&{1,188} \\ \hline
{Cores (Alerts) Annual refresh}&{}&{}&{1,188}&{}&{}&{1,188}&{}&{}&{1,188}& \\ \hline
{Cores (US DAC/Staff)}&{568}&{933}&{1,399}&{1,866}&{2,332}&{2,798}&{3,265}&{3,731}&{4,198}&{4,664} \\ \hline
{Cores (US DAC/Staff) Annual increase}&{428}&{364}&{466}&{466}&{466}&{466}&{466}&{466}&{466}&{466} \\ \hline
{Cores (US DAC/Staff) Annual refresh}&{}&{}&{141}&{428}&{364}&{466}&{466}&{466}&{466}&{466} \\ \hline
{Cores (US DAC/Staff) Annual purchase}&{428}&{364}&{607}&{894}&{831}&{933}&{933}&{933}&{933}&{933} \\ \hline
{Cores (Chilean DAC)}&{103}&{187}&{280}&{373}&{466}&{560}&{653}&{746}&{840}&{933} \\ \hline
{Cores (Chilean DAC) Annual increase}&{103}&{83}&{93}&{93}&{93}&{93}&{93}&{93}&{93}&{93} \\ \hline
{Cores (Chilean DAC) Annual refresh}&{}&{}&{0}&{103}&{83}&{93}&{93}&{93}&{93}&{93} \\ \hline
{Cores (Chilean DAC) Annual purchase}&{103}&{83}&{93}&{197}&{177}&{187}&{187}&{187}&{187}&{187} \\ \hline
{Qserv nodes (US DAC/Staff)}&{95}&{216}&{309}&{348}&{364}&{451}&{436}&{408}&{367}&{418} \\ \hline
{Qserv nodes (US DAC/Staff) Annual Increase}&{81}&{121}&{93}&{120}&{137}&{180}&{105}&{109}&{139}&{156} \\ \hline
{Qserv nodes (Chilean DAC)}&{95}&{216}&{309}&{348}&{364}&{451}&{436}&{408}&{367}&{418} \\ \hline
{Qserv nodes (Chilean DAC) Annual Increase}&{95}&{121}&{93}&{134}&{137}&{180}&{119}&{109}&{139}&{170} \\ \hline
\textbf{Total Cores Annual Increase}&\textbf{7,624}&\textbf{8,042}&\textbf{13,238}&\textbf{16,696}&\textbf{17,113}&\textbf{19,332}&\textbf{18,144}&\textbf{18,144}&\textbf{19,332}&\textbf{18,144} \\ \hline
{Fast Storage (TB)}&{206}&{371}&{586}&{667}&{735}&{798}&{859}&{918}&{974}&{1029} \\ \hline
{Annual Increase (Fast)}&{156}&{164}&{215}&{81}&{68}&{63}&{60}&{59}&{57}&{55} \\ \hline
{Annual Refresh (Fast)}&{}&{}&{}&{}&{26}&{156}&{164}&{215}&{81}&{68} \\ \hline
{Annual Purchase (Fast)}&{156}&{164}&{215}&{81}&{94}&{220}&{225}&{275}&{138}&{123} \\ \hline
{Normal Storage (TB)}&{24,081}&{39608}&{57486}&{75289}&{92901}&{110623}&{128499}&{146518}&{164631}&{182890} \\ \hline
{Annual Increase (Normal)}&{17,451}&{15527}&{17878}&{17803}&{17612}&{17722}&{17876}&{18019}&{18113}&{18259} \\ \hline
{Annual Refresh (Normal)}&{}&{}&{}&{}&{3,095}&{17,451}&{15,527}&{17,878}&{17,803}&{17,612} \\ \hline
{Annual Purchase (Normal)}&{17,451}&{15,527}&{17,878}&{17,803}&{20,706}&{35,173}&{33,404}&{35,898}&{35,915}&{35,871} \\ \hline
{Latent Storage  (TB)}&{16,617}&{35,478}&{57,062}&{77,209}&{97,356}&{117,503}&{137,650}&{157,797}&{177,944}&{198,091} \\ \hline
{Annual Increase (Latent)}&{13,188}&{18,860}&{21,584}&{20,147}&{20,147}&{20,147}&{20,147}&{20,147}&{20,147}&{20,147} \\ \hline
{Annual Refresh (Latent)}&{}&{}&{}&{}&{2,553}&{13,188}&{18,860}&{21,584}&{20,147}&{20,147} \\ \hline
{Annual Purchase (Latent)}&{13,188}&{18,860}&{21,584}&{20,147}&{22,700}&{33,335}&{39,008}&{41,731}&{40,294}&{40,294} \\ \hline
{High Latency (TB)}&{40,472}&{80,310}&{135,056}&{204,549}&{288,456}&{386,796}&{499,594}&{626,875}&{768,654}&{924,955} \\ \hline
{Annual Increase (High Latency)}&{27,256}&{39,837}&{54,746}&{69,493}&{83,907}&{98,340}&{112,798}&{127,281}&{141,779}&{156,301} \\ \hline
{Chilean DAC Fast Storage (TB)}&{267}&{435}&{622}&{795}&{937}&{1,080}&{1,222}&{1,364}&{1,507}&{1,649} \\ \hline
{Annual Increase (Fast Chilean DAC)}&{267}&{168}&{187}&{173}&{142}&{143}&{142}&{142}&{143}&{142} \\ \hline
{Annual Refresh (Fast Chilean DAC)}&{}&{}&{}&{}&{}&{267}&{168}&{187}&{173}&{142} \\ \hline
{Annual Purchase (Fast Chilean DAC)}&{267}&{168}&{187}&{173}&{142}&{410}&{310}&{329}&{316}&{284} \\ \hline
{Chilean DAC Latent Storage (TB)}&{10,500}&{18,391}&{25,950}&{31,007}&{36,063}&{41,120}&{46,177}&{51,234}&{56,291}&{61,348} \\ \hline
{Annual Increase (Latent Chilean DAC)}&{10,500}&{7,891}&{7,559}&{5,057}&{5,056}&{5,057}&{5,057}&{5,057}&{5,057}&{5,057} \\ \hline
{Annual Refresh (Latent Chilean DAC)}&{}&{}&{}&{}&{}&{10,500}&{7,891}&{7,559}&{5,057}&{5,056} \\ \hline
{Annual Purchase (Latent Chilean DAC)}&{10,500}&{7,891}&{7,559}&{5,057}&{5,056}&{15,557}&{12,948}&{12,616}&{10,114}&{10,113} \\ \hline
\end{longtable} \normalsize




