\subsection{Ops Cost details}\label{sec:opsdetails}
\tabref{tab:opsXeon} gives the price of compute based on Xeons.
This is broken down further for US in \tabref{tab:opsXeonUSDF} and Chile in \tabref{tab:opsXeonChile}.
However the Ops costing for SLAC was done using \tabref{tab:opsRomeUSDF}.

\tabref{tab:opsStorageCost} gives the price of storage using all  types that we need.
This is broken down further for US in \tabref{tab:opsStorageUSDF} and Chile in \tabref{tab:opsStorageChile}
This would be needed regardless of the compute chosen.
\tiny \begin{longtable} { |p{0.22\textwidth}  |r  |r  |r  |r  |r  |r  |r  |r  |r  |r  |r |} 
\caption{Implementation with Intel Xeon \label{tab:opsXeon}}\\ 
\hline 
\textbf{Year}&\textbf{2023}&\textbf{2024}&\textbf{2025}&\textbf{2026}&\textbf{2027}&\textbf{2028}&\textbf{2029}&\textbf{2030}&\textbf{2031}&\textbf{2032} \\ \hline
{Number of Xeon}&{282}&{298}&{470}&{618}&{634}&{694}&{672}&{672}&{694}&{672} \\ \hline
{Approximate cost (2019 Mdollars)}&{\$2.82}&{\$2.98}&{\$4.70}&{\$6.18}&{\$6.34}&{\$6.94}&{\$6.72}&{\$6.72}&{\$6.94}&{\$6.72} \\ \hline
\end{longtable} \normalsize

\tiny \begin{longtable} { |p{0.22\textwidth}  |r  |r  |r  |r  |r  |r  |r  |r  |r  |r  |r |} 
\caption{Implementation with Rome for USDF \label{tab:opsRomeUSDF}}\\ 
\hline 
\textbf{Year}&\textbf{2023}&\textbf{2024}&\textbf{2025}&\textbf{2026}&\textbf{2027}&\textbf{2028}&\textbf{2029}&\textbf{2030}&\textbf{2031}&\textbf{2032} \\ \hline
{Number of Large Rome USDF}&{65}&{69}&{114}&{143}&{147}&{166}&{155}&{155}&{166}&{155} \\ \hline
{Approximate cost (2019 Mdollars)}&{\$0.65}&{\$0.69}&{\$1.14}&{\$1.43}&{\$1.47}&{\$1.66}&{\$1.55}&{\$1.55}&{\$1.66}&{\$1.55} \\ \hline
\end{longtable} \normalsize

\tiny \begin{longtable} { |p{0.22\textwidth}  |r  |r  |r  |r  |r  |r  |r  |r  |r  |r  |r |} 
\caption{Implementation with Intel Xeon for USDF \label{tab:opsXeonUSDF}}\\ 
\hline 
\textbf{Year}&\textbf{2023}&\textbf{2024}&\textbf{2025}&\textbf{2026}&\textbf{2027}&\textbf{2028}&\textbf{2029}&\textbf{2030}&\textbf{2031}&\textbf{2032} \\ \hline
{Number of Xeon USDF}&{279}&{295}&{487}&{612}&{628}&{710}&{666}&{666}&{710}&{666} \\ \hline
{Approximate cost (2019 Mdollars)}&{\$2.79}&{\$2.95}&{\$4.87}&{\$6.12}&{\$6.28}&{\$7.10}&{\$6.66}&{\$6.66}&{\$7.10}&{\$6.66} \\ \hline
\end{longtable} \normalsize

\tiny \begin{longtable} { |p{0.22\textwidth}  |r  |r  |r  |r  |r  |r  |r  |r  |r  |r  |r |} 
\caption{Implementation with Intel Xeon for Chile Compute \label{tab:opsXeonChile}}\\ 
\hline 
\textbf{Year}&\textbf{2024}&\textbf{2025}&\textbf{2026}&\textbf{2027}&\textbf{2028}&\textbf{2029}&\textbf{2030}&\textbf{2031}&\textbf{2032}&\textbf{2033} \\ \hline
{Number of Xeon Chile}&{4}&{4}&{4}&{8}&{7}&{7}&{7}&{7}&{7}&{7} \\ \hline
{Approximate cost (2020 Mdollars)}&{\$0.04}&{\$0.04}&{\$0.04}&{\$0.08}&{\$0.07}&{\$0.07}&{\$0.07}&{\$0.07}&{\$0.07}&{\$0.07} \\ \hline
\end{longtable} \normalsize

\tiny \begin{longtable} { |p{0.22\textwidth}  |r  |r  |r  |r  |r  |r  |r  |r  |r  |r  |r |} 
\caption{Implementation with AMD Rome for Chile Compute \label{tab:opsRomeChile}}\\ 
\hline 
\textbf{Year}&\textbf{2024}&\textbf{2025}&\textbf{2026}&\textbf{2027}&\textbf{2028}&\textbf{2029}&\textbf{2030}&\textbf{2031}&\textbf{2032}&\textbf{2033} \\ \hline
{Number of Large Rome  Chile}&{3}&{3}&{3}&{6}&{6}&{6}&{6}&{6}&{6}&{6} \\ \hline
{Approximate cost (2020 Mdollars)}&{\$0.04}&{\$0.04}&{\$0.04}&{\$0.08}&{\$0.08}&{\$0.08}&{\$0.08}&{\$0.08}&{\$0.08}&{\$0.08} \\ \hline
\end{longtable} \normalsize

\tiny \begin{longtable} { |p{0.22\textwidth}  |r  |r  |r  |r  |r  |r  |r  |r  |r  |r  |r |} 
\caption{Total storage cost estimate for operations of Rubin Observatory USDF and CHile \label{tab:opsStorageCost}}\\ 
\hline 
\textbf{Year (all in M\$)}&\textbf{2024}&\textbf{2025}&\textbf{2026}&\textbf{2027}&\textbf{2028}&\textbf{2029}&\textbf{2030}&\textbf{2031}&\textbf{2032}&\textbf{2033} \\ \hline
{Fast Storage}&{\$0.20}&{\$0.16}&{\$0.19}&{\$0.13}&{\$0.12}&{\$0.31}&{\$0.27}&{\$0.30}&{\$0.23}&{\$0.22} \\ \hline
{Normal Storage}&{\$3.96}&{\$3.86}&{\$4.20}&{\$4.19}&{\$4.89}&{\$8.13}&{\$8.05}&{\$8.41}&{\$8.42}&{\$8.41} \\ \hline
{Latent Storage}&{\$5.36}&{\$6.44}&{\$7.00}&{\$5.39}&{\$5.94}&{\$10.75}&{\$11.83}&{\$12.39}&{\$10.78}&{\$10.78} \\ \hline
{High Latency Storage}&{\$0.73}&{\$1.13}&{\$1.56}&{\$2.00}&{\$2.43}&{\$2.86}&{\$3.28}&{\$3.72}&{\$4.15}&{\$4.58} \\ \hline
\textbf{Total (M\$)}&\textbf{\$10.24}&\textbf{\$11.58}&\textbf{\$12.95}&\textbf{\$11.70}&\textbf{\$13.38}&\textbf{\$22.04}&\textbf{\$23.44}&\textbf{\$24.82}&\textbf{\$23.58}&\textbf{\$23.99} \\ \hline
\end{longtable} \normalsize

\tiny \begin{longtable} { |p{0.22\textwidth}  |r  |r  |r  |r  |r  |r  |r  |r  |r  |r  |r |} 
\caption{Total storage cost estimate for operations at USDF \label{tab:opsStorageUSDF}}\\ 
\hline 
\textbf{Year (all in M\$)}&\textbf{2024}&\textbf{2025}&\textbf{2026}&\textbf{2027}&\textbf{2028}&\textbf{2029}&\textbf{2030}&\textbf{2031}&\textbf{2032}&\textbf{2033} \\ \hline
{Fast Storage USDF}&{\$0.06}&{\$0.07}&{\$0.09}&{\$0.03}&{\$0.04}&{\$0.09}&{\$0.09}&{\$0.11}&{\$0.06}&{\$0.05} \\ \hline
{Normal Storage USDF}&{\$3.96}&{\$3.86}&{\$4.20}&{\$4.19}&{\$4.89}&{\$8.13}&{\$8.05}&{\$8.41}&{\$8.42}&{\$8.41} \\ \hline
{Latent Storage USDF}&{\$3.18}&{\$4.69}&{\$5.37}&{\$4.72}&{\$5.26}&{\$7.90}&{\$9.40}&{\$10.09}&{\$9.44}&{\$9.44} \\ \hline
{High Latency Storage USDF}&{\$0.73}&{\$1.13}&{\$1.56}&{\$2.00}&{\$2.43}&{\$2.86}&{\$3.28}&{\$3.72}&{\$4.15}&{\$4.58} \\ \hline
\textbf{Total (M\$)}&\textbf{\$7.92}&\textbf{\$9.73}&\textbf{\$11.22}&\textbf{\$10.94}&\textbf{\$12.62}&\textbf{\$18.97}&\textbf{\$20.83}&\textbf{\$22.33}&\textbf{\$22.05}&\textbf{\$22.47} \\ \hline
\end{longtable} \normalsize

\tiny \begin{longtable} { |p{0.22\textwidth}  |r  |r  |r  |r  |r  |r  |r  |r  |r  |r  |r |} 
\caption{Total storage cost estimate for operations in Chile  {\bf Note} the latent storage here is 1.05 of the Raw and LSSTCam Coadd image volume. \label{tab:opsStorageChile}}\\ 
\hline 
\textbf{Year (all in M\$)}&\textbf{2024}&\textbf{2025}&\textbf{2026}&\textbf{2027}&\textbf{2028}&\textbf{2029}&\textbf{2030}&\textbf{2031}&\textbf{2032}&\textbf{2033} \\ \hline
{Fast Storage Chile}&{\$0.13}&{\$0.09}&{\$0.10}&{\$0.10}&{\$0.08}&{\$0.22}&{\$0.18}&{\$0.19}&{\$0.18}&{\$0.17} \\ \hline
{Latent Storage Chile}&{\$2.18}&{\$1.75}&{\$1.62}&{\$0.67}&{\$0.67}&{\$2.85}&{\$2.42}&{\$2.30}&{\$1.35}&{\$1.35} \\ \hline
\textbf{Total (M\$)}&\textbf{\$2.32}&\textbf{\$1.85}&\textbf{\$1.73}&\textbf{\$0.77}&\textbf{\$0.76}&\textbf{\$3.07}&\textbf{\$2.60}&\textbf{\$2.48}&\textbf{\$1.52}&\textbf{\$1.51} \\ \hline
\end{longtable} \normalsize


\tabref{tab:opsOverheadCost} gives the annual cost of hosting compute in NCSA. This includes purchasing racks to house
new nodes etc.
\tiny \begin{longtable} { |p{0.22\textwidth}  |r  |r  |r  |r  |r  |r  |r  |r  |r  |r  |r |} 
\caption{Overheads(NCSA) per year based on number of cores in \tabref{tab:opsInputs} and costs in \tabref{tab:overheads} assuming Xeon density from \tabref{tab:Machines}.  \label{tab:opsOverheadCost}}\\ 
\hline 
\textbf{Year}&\textbf{2023}&\textbf{2024}&\textbf{2025}&\textbf{2026}&\textbf{2027}&\textbf{2028}&\textbf{2029}&\textbf{2030}&\textbf{2031}&\textbf{2032} \\ \hline
\textbf{Total Incremental cores (USA)}&\textbf{7,521}&\textbf{7,958}&\textbf{8,979}&\textbf{8,979}&\textbf{8,979}&\textbf{8,979}&\textbf{8,979}&\textbf{8,979}&\textbf{8,979}&\textbf{8,979} \\ \hline
\textbf{Total owned cores (USA)}&\textbf{15,075}&\textbf{23,033}&\textbf{32,012}&\textbf{40,990}&\textbf{49,969}&\textbf{58,947}&\textbf{67,926}&\textbf{76,904}&\textbf{85,883}&\textbf{94,861} \\ \hline
\textbf{Total owned nodes}&\textbf{567}&\textbf{936}&\textbf{1,310}&\textbf{1,629}&\textbf{1,926}&\textbf{2,294}&\textbf{2,559}&\textbf{2,812}&\textbf{3,051}&\textbf{3,383} \\ \hline
{Cost for hosting nodes}&{\$320,801}&{\$529,576}&{\$741,180}&{\$921,666}&{\$1,089,704}&{\$1,297,914}&{\$1,447,847}&{\$1,590,991}&{\$1,726,214}&{\$1,914,055} \\ \hline
\textbf{Total new nodes}&\textbf{317}&\textbf{370}&\textbf{374}&\textbf{401}&\textbf{418}&\textbf{461}&\textbf{386}&\textbf{390}&\textbf{420}&\textbf{437} \\ \hline
\textbf{Total new racks}&\textbf{9}&\textbf{11}&\textbf{11}&\textbf{12}&\textbf{12}&\textbf{13}&\textbf{11}&\textbf{11}&\textbf{12}&\textbf{13} \\ \hline
{Rack install cost }&{\$216,000.00}&{\$264,000.00}&{\$264,000.00}&{\$288,000.00}&{\$288,000.00}&{\$312,000.00}&{\$264,000.00}&{\$264,000.00}&{\$288,000.00}&{\$312,000.00} \\ \hline
\textbf{Total Ops Overhead (NCSA)}&\textbf{\$536,800.80}&\textbf{\$793,575.92}&\textbf{\$1,005,179.97}&\textbf{\$1,209,665.78}&\textbf{\$1,377,704.29}&\textbf{\$1,609,913.63}&\textbf{\$1,711,846.98}&\textbf{\$1,854,990.90}&\textbf{\$2,014,213.81}&\textbf{\$2,226,054.84} \\ \hline
\end{longtable} \normalsize

{bf Note:} rack costs are for new racks so only paid for the added racks each year, hence some zeros appear when we do not intend to add racks.

For Chile \tabref{tab:opsOverheadChile} gives the cost of hosting in Chile (las Serena).
\tiny \begin{longtable} { |p{0.22\textwidth}  |r  |r  |r  |r  |r  |r  |r  |r  |r  |r  |r |} 
\caption{Overheads(Chile) per year based on number of cores in \tabref{tab:opsInputs} and costs in \tabref{tab:overheads} assuming Xeon density from \tabref{tab:Machines}.  \label{tab:opsOverheadChile}}\\ 
\hline 
\textbf{Year}&\textbf{2024}&\textbf{2025}&\textbf{2026}&\textbf{2027}&\textbf{2028}&\textbf{2029}&\textbf{2030}&\textbf{2031}&\textbf{2032}&\textbf{2033} \\ \hline
\textbf{Total Incremental cores (Chile)}&\textbf{103}&\textbf{83}&\textbf{93}&\textbf{93}&\textbf{93}&\textbf{93}&\textbf{93}&\textbf{93}&\textbf{93}&\textbf{93} \\ \hline
\textbf{Total owned cores (Chile)}&\textbf{103}&\textbf{187}&\textbf{280}&\textbf{373}&\textbf{466}&\textbf{560}&\textbf{653}&\textbf{746}&\textbf{840}&\textbf{933} \\ \hline
{Compute nodes}&{4}&{6}&{9}&{12}&{15}&{18}&{21}&{24}&{27}&{30} \\ \hline
{Qserv nodes}&{95}&{216}&{309}&{348}&{364}&{451}&{436}&{408}&{367}&{418} \\ \hline
\textbf{Total Nodes}&\textbf{99}&\textbf{222}&\textbf{318}&\textbf{360}&\textbf{379}&\textbf{469}&\textbf{457}&\textbf{432}&\textbf{394}&\textbf{448} \\ \hline
\textbf{Total Compute Racks}&\textbf{3}&\textbf{7}&\textbf{9}&\textbf{10}&\textbf{11}&\textbf{14}&\textbf{13}&\textbf{12}&\textbf{11}&\textbf{13} \\ \hline
\textbf{Total Storage}&\textbf{16,494}&\textbf{13,387}&\textbf{12,525}&\textbf{5,404}&\textbf{5,421}&\textbf{21,907}&\textbf{18,664}&\textbf{17,681}&\textbf{10,480}&\textbf{10,531} \\ \hline
\textbf{Total Storage Racks}&\textbf{3}&\textbf{2}&\textbf{2}&\textbf{1}&\textbf{1}&\textbf{3}&\textbf{3}&\textbf{3}&\textbf{2}&\textbf{2} \\ \hline
{Cooling Power Kw}&{12.00}&{18.00}&{22.00}&{22.00}&{24.00}&{34.00}&{32.00}&{30.00}&{26.00}&{30.00} \\ \hline
{Computing Power kW
}&{87.6}&{131.4}&{160.6}&{160.6}&{175.2}&{248.2}&{233.6}&{219}&{189.8}&{219} \\ \hline
{Power Cost}&{\$131,748}&{\$197,622}&{\$241,538}&{\$241,538}&{\$263,496}&{\$373,286}&{\$351,328}&{\$329,370}&{\$285,454}&{\$329,370} \\ \hline
{Compute rack install costs}&{\$85,902.00}&{\$114,536.00}&{\$57,268.00}&{\$28,634.00}&{\$28,634.00}&{\$85,902.00}&{\$0.00}&{\$0.00}&{\$0.00}&{\$0.00} \\ \hline
{Storage rack install costs}&{\$85,902.00}&{\$0.00}&{\$0.00}&{\$0.00}&{\$0.00}&{\$57,268.00}&{\$0.00}&{\$0.00}&{\$0.00}&{\$0.00} \\ \hline
\textbf{Total Ops Overhead Chile (USD)}&\textbf{\$303,552}&\textbf{\$312,158}&\textbf{\$298,806}&\textbf{\$270,172}&\textbf{\$292,130}&\textbf{\$516,456}&\textbf{\$351,328}&\textbf{\$329,370}&\textbf{\$285,454}&\textbf{\$329,370} \\ \hline
\end{longtable} \normalsize


For Chile the rack costs are outlined in \tabref{tab:rackCostChile}.
\tiny \begin{longtable} {{l | l | r| r |r|r| l}} \caption{This table details the cost per rack which is added in \tabref{tab:opsOverheadChile}. \label{tab:rackCostChile}}\\ 
\hline 
\textbf{Rack}&\textbf{}&\textbf{Unit Cost}&\textbf{Spine Port}&\textbf{}&\textbf{Total}& \\ \hline
{2020 Rack Component }&{2 x Leaf}&{\$7,000.00}&{\$2,187.00}&{\$9,187.00}&{\$18,374.00}&{Cisco Nexus 93108TC-EX, + overhead of Spine port} \\ \hline
{}&{2 x PDU}&{\$2,980.00}&{}&{}&{\$5,960.00}&{Raritan PX3 5085U-N2} \\ \hline
{}&{1 x IPMI}&{\$1,800.00}&{}&{}&{\$1,800.00}&{Cisco Catalyst 2960-X /  Cisco 9200} \\ \hline
{}&{1 x Rack}&{\$2,500.00}&{}&{}&{\$2,500.00}&{APC AR3357} \\ \hline
\textbf{Total}&\textbf{}&\textbf{}&\textbf{}&\textbf{}&\textbf{\$28,634.00}& \\ \hline
\end{longtable} \normalsize


For Chile the power costs are outlined in \tabref{tab:powerCostChile}.
\tiny \begin{longtable} {{ | l | r| r |r|r| r |}} \caption{Cost is estimated to increase 5-10\% every 2-3 years \label{tab:powerCostChile}}\\ 
\hline 
\textbf{Year}&\textbf{2021}&\textbf{2024}&\textbf{2027}&\textbf{2030}&\textbf{2033} \\ \hline
{Power Cost}&{100.21}&{110.231}&{121.2541}&{133.37951}&{146.717461} \\ \hline
\end{longtable} \normalsize


Various other inputs to ops costing are given in \tabref{tab:opsInputs}.
\tiny \begin{longtable} { |p{0.22\textwidth}  |r  |r  |r  |r  |r  |r  |r  |r  |r  |r  |r |} 
\caption{Various inputs for deriving costs in operations - these drive the costs in \tabref{tab:opsSummary}. This is based on \tabref{tab:opsXeon}, \tabref{tab:opsStorageCost}  \label{tab:opsInputs}}\\ 
\hline 
\textbf{Year}&\textbf{2024}&\textbf{2025}&\textbf{2026}&\textbf{2027}&\textbf{2028}&\textbf{2029}&\textbf{2030}&\textbf{2031}&\textbf{2032}&\textbf{2033} \\ \hline
{Core-hours Needed Total (DRP)}&{4.5E+07}&{8.2E+07}&{1.2E+08}&{1.6E+08}&{2.0E+08}&{2.5E+08}&{2.9E+08}&{3.3E+08}&{3.7E+08}&{4.1E+08} \\ \hline
{Core-hours Annual Increase}&{3.40E+07}&{3.6E+07}&{4.1E+07}&{4.1E+07}&{4.1E+07}&{4.1E+07}&{4.1E+07}&{4.1E+07}&{4.1E+07}&{4.1E+07} \\ \hline
{Time to Process days}&{200}&{200}&{200}&{200}&{200}&{200}&{200}&{200}&{200}&{200} \\ \hline
{Time to Process hours}&{4,800}&{4,800}&{4,800}&{4,800}&{4,800}&{4,800}&{4,800}&{4,800}&{4,800}&{4,800} \\ \hline
{Cores (DRP) Annual increase}&{7,093}&{7,594}&{8,512}&{8,512}&{8,512}&{8,512}&{8,512}&{8,512}&{8,512}&{8,512} \\ \hline
{Cores (DRP) Annual refresh}&{}&{}&{2,837}&{7,093}&{7,594}&{8,512}&{8,512}&{8,512}&{8,512}&{8,512} \\ \hline
{Cores (DRP) Annual purchase}&{7,093}&{7,594}&{11,349}&{15,605}&{16,106}&{17,024}&{17,024}&{17,024}&{17,024}&{17,024} \\ \hline
{Cores (Alerts)}&{1,188}&{1,188}&{1,188}&{1,188}&{1,188}&{1,188}&{1,188}&{1,188}&{1,188}&{1,188} \\ \hline
{Cores (Alerts) Annual refresh}&{}&{}&{1,188}&{}&{}&{1,188}&{}&{}&{1,188}& \\ \hline
{Cores (US DAC/ Staff)}&{568}&{933}&{1,399}&{1,866}&{2,332}&{2,798}&{3,265}&{3,731}&{4,198}&{4,664} \\ \hline
{Cores (US DAC/ Staff) Annual increase}&{428}&{364}&{466}&{466}&{466}&{466}&{466}&{466}&{466}&{466} \\ \hline
{Cores (US DAC/ Staff) Annual refresh}&{}&{}&{141}&{428}&{364}&{466}&{466}&{466}&{466}&{466} \\ \hline
{Cores (US DAC/ Staff) Annual purchase}&{428}&{364}&{607}&{894}&{831}&{933}&{933}&{933}&{933}&{933} \\ \hline
{Cores (Chilean DAC)}&{103}&{187}&{280}&{373}&{466}&{560}&{653}&{746}&{840}&{933} \\ \hline
{Cores (Chilean DAC) Annual increase}&{103}&{83}&{93}&{93}&{93}&{93}&{93}&{93}&{93}&{93} \\ \hline
{Cores (Chilean DAC) Annual refresh}&{}&{}&{0}&{103}&{83}&{93}&{93}&{93}&{93}&{93} \\ \hline
{Cores (Chilean DAC) Annual purchase}&{103}&{83}&{93}&{197}&{177}&{187}&{187}&{187}&{187}&{187} \\ \hline
{Qserv nodes (US DAC/ Staff)}&{95}&{216}&{309}&{348}&{364}&{451}&{436}&{408}&{367}&{418} \\ \hline
{Qserv nodes (US DAC/ Staff) Annual Increase}&{81}&{121}&{93}&{120}&{137}&{180}&{105}&{109}&{139}&{156} \\ \hline
{Qserv nodes (Chilean DAC)}&{95}&{216}&{309}&{348}&{364}&{451}&{436}&{408}&{367}&{418} \\ \hline
{Qserv nodes (Chilean DAC) Annual Increase}&{95}&{121}&{93}&{134}&{137}&{180}&{119}&{109}&{139}&{170} \\ \hline
\textbf{Total Cores Annual Increase}&\textbf{7,624}&\textbf{8,042}&\textbf{13,238}&\textbf{16,696}&\textbf{17,113}&\textbf{19,332}&\textbf{18,144}&\textbf{18,144}&\textbf{19,332}&\textbf{18,144} \\ \hline
{Fast Storage (TB)}&{206}&{371}&{586}&{667}&{735}&{798}&{859}&{918}&{974}&{1029} \\ \hline
{Annual Increase (Fast)}&{156}&{164}&{215}&{81}&{68}&{63}&{60}&{59}&{57}&{55} \\ \hline
{Annual Refresh (Fast)}&{}&{}&{}&{}&{26}&{156}&{164}&{215}&{81}&{68} \\ \hline
{Annual Purchase (Fast)}&{156}&{164}&{215}&{81}&{94}&{220}&{225}&{275}&{138}&{123} \\ \hline
{Normal Storage (TB)}&{38,983}&{67982}&{99544}&{131031}&{162327}&{193733}&{225294}&{256997}&{288794}&{320737} \\ \hline
{Annual Increase (Normal)}&{29,742}&{28999}&{31563}&{31487}&{31296}&{31406}&{31560}&{31703}&{31797}&{31943} \\ \hline
{Annual Refresh (Normal)}&{}&{}&{}&{}&{5,494}&{29,742}&{28,999}&{31,563}&{31,487}&{31,296} \\ \hline
{Annual Purchase (Normal)}&{29,742}&{28,999}&{31,563}&{31,487}&{36,790}&{61,148}&{60,559}&{63,266}&{63,284}&{63,239} \\ \hline
{Latent Storage  (TB)}&{28,854}&{64,086}&{104,491}&{139,969}&{175,447}&{210,925}&{246,403}&{281,881}&{317,359}&{352,837} \\ \hline
{Annual Increase (Latent)}&{23,888}&{35,232}&{40,405}&{35,478}&{35,478}&{35,478}&{35,478}&{35,478}&{35,478}&{35,478} \\ \hline
{Annual Refresh (Latent)}&{}&{}&{}&{}&{4,090}&{23,888}&{35,232}&{40,405}&{35,478}&{35,478} \\ \hline
{Annual Purchase (Latent)}&{23,888}&{35,232}&{40,405}&{35,478}&{39,568}&{59,366}&{70,710}&{75,884}&{70,956}&{70,956} \\ \hline
{High Latency (TB)}&{63,245}&{135,135}&{234,943}&{362,508}&{517,497}&{699,929}&{909,829}&{1,147,221}&{1,412,122}&{1,704,554} \\ \hline
{Annual Increase (High Latency)}&{46,512}&{71,890}&{99,809}&{127,565}&{154,989}&{182,432}&{209,900}&{237,392}&{264,900}&{292,432} \\ \hline
{Chilean DAC Fast Storage (TB)}&{156}&{347}&{562}&{643}&{711}&{774}&{835}&{894}&{951}&{1,006} \\ \hline
{Annual Increase (Fast Chilean DAC)}&{156}&{190}&{215}&{81}&{68}&{63}&{60}&{59}&{57}&{55} \\ \hline
{Annual Refresh (Fast Chilean DAC)}&{}&{}&{}&{}&{}&{156}&{190}&{215}&{81}&{68} \\ \hline
{Annual Purchase (Fast Chilean DAC)}&{156}&{190}&{215}&{81}&{68}&{220}&{251}&{275}&{138}&{123} \\ \hline
{Chilean DAC Latent Storage (TB)}&{28,854}&{64,086}&{104,491}&{139,969}&{175,447}&{210,925}&{246,403}&{281,881}&{317,359}&{352,837} \\ \hline
{Annual Increase (Latent Chilean DAC)}&{28,854}&{35,232}&{40,405}&{35,478}&{35,478}&{35,478}&{35,478}&{35,478}&{35,478}&{35,478} \\ \hline
{Annual Refresh (Latent Chilean DAC)}&{}&{}&{}&{}&{}&{28,854}&{35,232}&{40,405}&{35,478}&{35,478} \\ \hline
{Annual Purchase (Latent Chilean DAC)}&{28,854}&{35,232}&{40,405}&{35,478}&{35,478}&{64,332}&{70,710}&{75,884}&{70,956}&{70,956} \\ \hline
\end{longtable} \normalsize



\subsubsection{Alternative costing for Chile}
The alternative cost for Chile including Xeons and Qserv but with a similar storage model is given in \tabref{tab:opsSumChile}.
\tiny \begin{longtable} { |p{0.22\textwidth}  |r  |r  |r  |r  |r  |r  |r  |r  |r  |r  |r |} 
\caption{This table pulls together all the information in a high level summary for Chile operations - in this table Xeon pricing(see \tabref{tab:opsXeonChile}) is used since that is the more expensive but better known option. Price factors, defined in \tabref{tab:Inputs} are applied in all cases - other input values come from \tabref{tab:opsInputs}, \tabref{tab:opsStorageChile}.
 \label{tab:opsSumChile}}\\ 
\hline 
\textbf{Year  (all prices Million\$)}&\textbf{2024}&\textbf{2025}&\textbf{2026}&\textbf{2027}&\textbf{2028}&\textbf{2029}&\textbf{2030}&\textbf{2031}&\textbf{2032}&\textbf{2033} \\ \hline
{Compute (2019 pricing)}&{\$0.04}&{\$0.04}&{\$0.04}&{\$0.08}&{\$0.07}&{\$0.07}&{\$0.07}&{\$0.07}&{\$0.07}&{\$0.07} \\ \hline
{Qserv (2019 pricing)}&{\$1.90}&{\$2.42}&{\$1.86}&{\$2.68}&{\$2.74}&{\$3.60}&{\$2.38}&{\$2.18}&{\$2.78}&{\$3.40} \\ \hline
{Storage (2019 pricing)}&{\$3.90}&{\$4.76}&{\$5.46}&{\$4.75}&{\$4.75}&{\$8.64}&{\$9.50}&{\$10.20}&{\$9.49}&{\$9.49} \\ \hline
\textbf{Total (2019 pricing)}&\textbf{\$5.84}&\textbf{\$7.22}&\textbf{\$7.36}&\textbf{\$7.51}&\textbf{\$7.56}&\textbf{\$12.31}&\textbf{\$11.95}&\textbf{\$12.45}&\textbf{\$12.34}&\textbf{\$12.96} \\ \hline
{Applying price factor (CPU)}&{\$0.03}&{\$0.02}&{\$0.02}&{\$0.04}&{\$0.03}&{\$0.03}&{\$0.02}&{\$0.02}&{\$0.02}&{\$0.02} \\ \hline
{Qserv (applying factor)}&{\$1.32}&{\$1.53}&{\$1.07}&{\$1.41}&{\$1.32}&{\$1.58}&{\$0.95}&{\$0.80}&{\$0.93}&{\$1.03} \\ \hline
{Applying price factor (Storage)}&{\$2.86}&{\$3.22}&{\$3.42}&{\$2.75}&{\$2.54}&{\$4.29}&{\$4.36}&{\$4.33}&{\$3.72}&{\$3.44} \\ \hline
\textbf{Total budget (using price factors)}&\textbf{\$4.20}&\textbf{\$4.78}&\textbf{\$4.52}&\textbf{\$4.20}&\textbf{\$3.89}&\textbf{\$5.89}&\textbf{\$5.34}&\textbf{\$5.15}&\textbf{\$4.67}&\textbf{\$4.49} \\ \hline
\textbf{Total Operations hardware to 2032 }&\textbf{\$47.13}&\textbf{million}&&&&&&&& \\ \hline
\end{longtable} \normalsize

