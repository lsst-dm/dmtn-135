\documentclass[DM,authoryear,toc]{lsstdoc}
% lsstdoc documentation: https://lsst-texmf.lsst.io/lsstdoc.html
\input{meta}

% Package imports go here.

% Local commands go here.

%If you want glossaries
%\input{aglossary.tex}
%\makeglossaries

\title{DM sizing model and purchase plan for the remainder of construction.}

% Optional subtitle
% \setDocSubtitle{A subtitle}

\author{%
William O'Mullane
}

\setDocRef{DMTN-135}
\setDocUpstreamLocation{\url{https://github.com/lsst-dm/dmtn-135}}

\date{\vcsDate}

% Optional: name of the document's curator
% \setDocCurator{The Curator of this Document}

\setDocAbstract{%
This simplified model is based on real machine architectures we have today. We define the needs for commissioning and separately identify the DR1,2 needs which could be moved to operations. This is presented in s series of tables within this document describing the approach taken.
}

% Change history defined here.
% Order: oldest first.
% Fields: VERSION, DATE, DESCRIPTION, OWNER NAME.
% See LPM-51 for version number policy.
\setDocChangeRecord{%
  \addtohist{1}{YYYY-MM-DD}{Unreleased.}{William O'Mullane}
}


\begin{document}

% Create the title page.
\maketitle
% Frequently for a technote we do not want a title page  uncomment this to remove the title page and changelog.
% use \mkshorttitle to remove the extra pages

% ADD CONTENT HERE
% You can also use the \input command to include several content files.

\appendix
% Include all the relevant bib files.
% https://lsst-texmf.lsst.io/lsstdoc.html#bibliographies
\section{References} \label{sec:bib}
\bibliography{local,lsst,lsst-dm,refs_ads,refs,books}

% Make sure lsst-texmf/bin/generateAcronyms.py is in your path
\section{Acronyms} \label{sec:acronyms}
\addtocounter{table}{-1}
\begin{longtable}{p{0.145\textwidth}p{0.8\textwidth}}\hline
\textbf{Acronym} & \textbf{Description}  \\\hline

AMD & Advanced Micro Devices \\\hline
AP & Alert Production \\\hline
APDB & Alert Production DataBase \\\hline
CCD & Charge-Coupled Device \\\hline
CLP & Chilean Peso \\\hline
CPU & Central Processing Unit \\\hline
ComCam & The commissioning camera is a single-raft, 9-CCD camera that will be installed in LSST during commissioning, before the final camera is ready. \\\hline
DAC & Data Access Center \\\hline
DC2 & Data Challenge 2 (DESC) \\\hline
DCR & Differential Chromatic Refraction \\\hline
DDN & Data Delivery Network \\\hline
DES & Dark Energy Survey \\\hline
DESC & Dark Energy Science Collaboration \\\hline
DM & Data Management \\\hline
DMTN & DM Technical Note \\\hline
DP & Data Production \\\hline
DPDD & Data Product Definition Document \\\hline
DR1 & Data Release 1 \\\hline
DRP & Data Release Production \\\hline
FLOP & FLoating point Operation \\\hline
FLOPS & FLoating point Operation per Second \\\hline
FY21 & Financial Year 21 \\\hline
GB & Gigabyte \\\hline
GFLOPS & Giga FLOP per Second \\\hline
GPFS & General Parallel File System (now IBM Spectrum Scale) \\\hline
HSC & Hyper Suprime-Cam \\\hline
IN2P3 & Institut National de Physique Nucléaire et de Physique des Particules \\\hline
KW & Kilowatt \\\hline
LATISS & LSST Atmospheric Transmission Imager and Slitless Spectrograph \\\hline
LCR & LSST Change Request \\\hline
LDF & LSST Data Facility \\\hline
LDM & LSST Data Management (Document Handle) \\\hline
LSE & LSST Systems Engineering (Document Handle) \\\hline
LSP & LSST Science Platform (now Rubin Science Platform) \\\hline
LSST & Legacy Survey of Space and Time (formerly Large Synoptic Survey Telescope) \\\hline
MB & MegaByte \\\hline
MBTU & Mega British Thermal Unit \\\hline
MOF & Multi-Object Multi-Band Fitting \\\hline
NCSA & National Center for Supercomputing Applications \\\hline
NSF & National Science Foundation \\\hline
NVMe & Non Volatile Memory Express \\\hline
PB & PetaByte \\\hline
PCI & Peripheral Component Interconnect \\\hline
PDR & Preliminary Design Review \\\hline
PDR1 & Public Data Release 1 (HSC) \\\hline
PDR2 & Public Data Release 2 (HSC) \\\hline
QA & Quality Assurance \\\hline
RAM & Random Access Memory \\\hline
RFC & Request For Comment \\\hline
S3 & (Amazon) Simple Storage Service  \\\hline
SATA & Serial Advanced Technology Attachment \\\hline
SLAC & SLAC National Accelerator Laboratory \\\hline
SOF & Single-Object Fitting \\\hline
SQuaSH & Science Quality Analysis Harness \\\hline
SSD & Solid-State Disk \\\hline
SSP & Solar System Processing \\\hline
TB & TeraByte \\\hline
UKDF & United Kingdom Data Facility \\\hline
US & United States \\\hline
USDF & United States Data Facility \\\hline
VM & Virtual Machine \\\hline
deg & degree; unit of angle \\\hline
\end{longtable}

% If you want glossary uncomment below -- comment out the two lines above
%\printglossaries





\end{document}
