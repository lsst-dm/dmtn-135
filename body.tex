% Tables from https://docs.google.com/spreadsheets/d/1DiFTjsC4dP8XyOV7-uF0zwkl0r0jMuW9U9uELejpmn8/edit#gid=0
\section{Introduction}
This document presents the simplified
  sizing model for Rubin Observatory data management  in \secref{sec:sizemodel} based on detailed sizing presented in \secref{sec:sizeinputs}.
\secref{sec:cost} presents a very high level budget summary for
DM hardware which was used for \href{https://project.lsst.org/groups/ccb/node/3889}{LCR-2148}.
More interesting now is the build up at the USDF in pre-operations which is shown in \secref{sec:preopscost}
and the full operations estimates in \secref{sec:opscost}

This version is in agreement with SLAC on the parameters for CPU and disk price fall as well as CPU cost etc.

\section{Construction  budget}\label{sec:cost}

A high level bottom line is given in \tabref{tab:Summary}.
The remainder of the document is all the details that went into that.

\tiny \begin{longtable} { |p{0.22\textwidth}  |r  |r  |r  |r  |r |} 
\caption{This table pulls together all the information in a high level summary - in this table Xeon pricing is used since that is the more expensive but better known option. Price factors, defined in \tabref{tab:Inputs} are applied post 2020.
 \label{tab:Summary}}\\ 
\hline 
\textbf{Year}&\textbf{2020}&\textbf{2021}&\textbf{2022}&\textbf{2023} \\ \hline
{Compute (2019 pricing)}&{\$690,000}&{\$0}&{\$1,510,000}&{\$2,820,000} \\ \hline
{Storage (2019 pricing)}&{\$190,702}&{\$126,563}&{\$1,216,867}&{\$7,864,125} \\ \hline
{Qserv (2019 pricing)}&{}&{}&{\$560,000}&{\$3,240,000} \\ \hline
\textbf{Total (2019 pricing)}&\textbf{\$880,702}&\textbf{\$126,563}&\textbf{\$3,286,867}&\textbf{\$13,924,125} \\ \hline
{Compute (applying price factor)}&{\$621,000}&{\$0}&{\$1,057,000}&{\$1,692,000} \\ \hline
{IN2P3 (50\% of compute in ops)}&{}&{}&{}&{-\$846,000} \\ \hline
{Storage (applying price factor)}&{\$181,167}&{\$113,907}&{\$1,034,337}&{\$6,291,300} \\ \hline
{Qserv (applying price factor)}&{}&{}&{\$434,000}&{\$2,268,000} \\ \hline
{Hosting cost NCSA
}&{\$110,802}&{\$62,802}&{\$238,012}&{\$536,801} \\ \hline
\textbf{Total budget (using price factors)}&\textbf{\$912,969}&\textbf{\$176,709}&\textbf{\$2,763,350}&\textbf{\$9,942,100} \\ \hline
\end{longtable} \normalsize


We have applied a modest cost reduction assuming that processors  and disks get a little cheaper - that
percentage is given in \tabref{tab:Inputs} along with many other parameters. \tabref{tab:Inputs} also contains the number of nodes we assume to need for Qserv.

Specific  costs for storage are detailed in \tabref{tab:Storage} and for compute in \tabref {tab:Machines}
the following budgets can be considered. The detailed annual purchasing based on those prices is given for
storage in \tabref{tab:StorageCost} and for compute in \tabref{tab:Rome}.

\subsection{User compute}
In these tables there is a 10\%  of the USDF planned compute added as user compute.
The use of this compute was intended for user batch while the Science Platform did not exist when the number was imagined.
Hence the Science Platform compute must be considered part of this 10\%.
Whether that is sufficient or not is not discussed in this document.

\section{Pre -Operations budget estimate}\label{sec:opscost}
In this section we estimate the ramp up of the USDF to be ready for start of operations.
This means having compute for commissioning data as well as developer services and a small science platform in place.
It is very similar to the construction needs. The summary is in \tabref{tab:preOps}.

\tiny \begin{longtable} {|l |r  |r  |r |} \caption{This table builds a ramp for build up at SLAC as USDF. These would be purchases to get initial systems in place for the first year of operations. This is based on the Rome processor price and other construction inputs. \label{tab:preOps}}\\ 
\hline 
\textbf{Year (Pricing \$million)}&\textbf{2021}&\textbf{2022}&\textbf{2023} \\ \hline
{Compute (2020 pricing)}&{\$0.02}&{\$0.04}&{\$0.43} \\ \hline
{Qserv (2020 pricing)}&{}&{}&{\$0.28} \\ \hline
{Storage (2020 pricing)}&{\$0.31}&{\$0.10}&{\$0.52} \\ \hline
\textbf{Total (2020 pricing)}&\textbf{\$0.33}&\textbf{\$0.14}&\textbf{\$1.23} \\ \hline
{Applying price factor (CPU)}&{\$0.02}&{\$0.03}&{\$0.31} \\ \hline
{Qserv (applying factor)}&{0}&{0}&{\$0.22} \\ \hline
{Applying price factor (Storage)}&{\$0.30}&{\$0.09}&{\$0.45} \\ \hline
{Hosting Overhead SLAC}&{\$0.12}&{\$0.11}&{\$0.15} \\ \hline
\textbf{Total budget (using price factors)}&\textbf{\$0.43}&\textbf{\$0.22}&\textbf{\$1.13} \\ \hline
\textbf{Total Pre Ops hardware to 2023}&\textbf{\$1.79}&\textbf{million}& \\ \hline
\end{longtable} \normalsize


Currently we assume exactly the construction profile for storage. The compute is a little different and uses Rome
which is captured in \tabref{tab:preCompute}.

\tiny \begin{longtable} { |p{0.22\textwidth}  |r  |r  |r  |r |} 
\caption{Preoperations compute build up at USDF, cores we need to purchase per year. \label{tab:preCompute}}\\ 
\hline 
\textbf{Year}&\textbf{2021}&\textbf{2022}&\textbf{2023} \\ \hline
{DRP cores (from construction)}&{0}&{1,836}&{2,837} \\ \hline
{Alerts cores (from construction)}&{}&{}&{1188} \\ \hline
{Dev Cores}&{100}&{440}& \\ \hline
{K8S (science platform)}&{100}&{}&{924} \\ \hline
\textbf{Total cores}&\textbf{200}&\textbf{440}&\textbf{4,950} \\ \hline
{Number of Large Rome USDF}&{2}&{4}&{43} \\ \hline
{Compute (2020 pricing)}&{\$0.02}&{\$0.04}&{\$0.43} \\ \hline
\end{longtable} \normalsize



\section{Operations budget estimate}\label{sec:opscost}
Based on the needs in \tabref{tab:opsInputs} and the costs in \tabref{tab:opsStorageCost} and \tabref {tab:Machines}
we get the estimate presented in \tabref{tab:opsSummary}.
In \tabref{tab:opsSummary} we should note that IN2P3 do 50\%  and UKDF do 25\% of the processing so we reduce the processing cost by three quarters.
This does not reduce the storage cost.

\tiny \begin{longtable} { |p{0.22\textwidth}  |r  |r  |r  |r  |r  |r  |r  |r  |r  |r  |r |} 
\caption{This table pulls together all the information in a high level summary for operations  for the Chile and USDF. Price factors, defined in \tabref{tab:Inputs} are applied in all cases - other input values come from \tabref{tab:opsInputs}, \tabref{tab:opsStorageCost}.
 \label{tab:opsSummary}}\\ 
\hline 
\textbf{Year  (all prices Million\$)}&\textbf{2023}&\textbf{2024}&\textbf{2025}&\textbf{2026}&\textbf{2027}&\textbf{2028}&\textbf{2029}&\textbf{2030}&\textbf{2031}&\textbf{2032} \\ \hline
{Compute (2019 pricing)}&{\$0.65}&{\$0.69}&{\$1.14}&{\$1.43}&{\$1.47}&{\$1.66}&{\$1.55}&{\$1.55}&{\$1.66}&{\$1.55} \\ \hline
{Qserv (2019 pricing)}&{\$3.52}&{\$4.84}&{\$3.72}&{\$5.08}&{\$5.48}&{\$7.20}&{\$4.48}&{\$4.36}&{\$5.56}&{\$6.52} \\ \hline
{Storage (2019 pricing)}&{\$11.82}&{\$14.49}&{\$16.68}&{\$15.69}&{\$17.36}&{\$27.62}&{\$30.34}&{\$32.53}&{\$31.55}&{\$31.96} \\ \hline
\textbf{Total (2019 pricing)}&\textbf{\$15.99}&\textbf{\$20.02}&\textbf{\$21.54}&\textbf{\$22.20}&\textbf{\$24.31}&\textbf{\$36.48}&\textbf{\$36.37}&\textbf{\$38.44}&\textbf{\$38.77}&\textbf{\$40.03} \\ \hline
{Applying price factor (CPU)}&{\$0.43}&{\$0.41}&{\$0.61}&{\$0.68}&{\$0.63}&{\$0.64}&{\$0.54}&{\$0.49}&{\$0.47}&{\$0.39} \\ \hline
{IN2P3 (50\% of US compute)}&{-\$0.21}&{-\$0.20}&{-\$0.30}&{-\$0.34}&{-\$0.32}&{-\$0.32}&{-\$0.27}&{-\$0.24}&{-\$0.23}&{-\$0.20} \\ \hline
{UKDF (25\% of compute)}&{-\$0.11}&{-\$0.10}&{-\$0.15}&{-\$0.17}&{-\$0.16}&{-\$0.16}&{-\$0.14}&{-\$0.12}&{-\$0.12}&{-\$0.10} \\ \hline
{Qserv (applying factor)}&{\$2.44}&{\$3.06}&{\$2.15}&{\$2.68}&{\$2.63}&{\$3.16}&{\$1.79}&{\$1.59}&{\$1.85}&{\$1.98} \\ \hline
{Applying price factor (Storage)}&{\$8.66}&{\$9.82}&{\$10.45}&{\$9.09}&{\$9.31}&{\$13.69}&{\$13.91}&{\$13.80}&{\$12.38}&{\$11.60} \\ \hline
{Hosting Overhead SLAC}&{\$0.45}&{\$0.51}&{\$0.51}&{\$0.49}&{\$0.51}&{\$0.66}&{\$0.63}&{\$0.62}&{\$0.58}&{\$0.56} \\ \hline
\textbf{Total budget (using price factors)}&\textbf{\$11.65}&\textbf{\$13.49}&\textbf{\$13.26}&\textbf{\$12.42}&\textbf{\$12.61}&\textbf{\$17.67}&\textbf{\$16.47}&\textbf{\$16.13}&\textbf{\$14.93}&\textbf{\$14.24} \\ \hline
\textbf{Total Operations hardware to 2032 }&\textbf{\$142.88}&\textbf{million}&&&&&&&& \\ \hline
\end{longtable} \normalsize


Again in \tabref{tab:opsSummary} we assume IN2P3 do 50\% of processing.
We have applied a compounded modest cost reduction assuming that processors  and disks get a little cheaper - that
percentage is given in \tabref{tab:Inputs}.

It must be noted that the price of disk and tape have a profound effect over 10 years. We have been fairly conservative on the base prices
in \tabref{tab:Storage}. An even bigger effect is in the compounding of the presumed fall in storage cost. Here we have used an extremely
conservative 5\% per year (\tabref{tab:Inputs}) - changing this to 15\% halves the cumulative ops estimate, setting it to 10\% brings the
total down by about 30\%.

\subsection{Cloud costs}
In addition there are some cloud costs. We run certain jobs and host websites on Amazon and Google. In operations
the validation team may also wish to run simulations on cloud resources. This estimate is in \tabref{tab:cloud}.

\tiny \begin{longtable} { |p{0.22\textwidth}  |r  |r  |r  |r  |r  |r  |r  |r  |r  |r  |r  |r |} 
\caption{We have on going cloud costs and assume some other activities may be on cloud in the future - we make an estimate of those costs here. \label{tab:cloud}}\\ 
\hline 
\textbf{Year  (all prices Million\$)}&\textbf{2024}&\textbf{2025}&\textbf{2026}&\textbf{2027}&\textbf{2028}&\textbf{2029}&\textbf{2030}&\textbf{2031}&\textbf{2032}&\textbf{2033}&\textbf{2034} \\ \hline
{Jira Cloud}&{\$80,000}&{\$80,000}&{\$80,000}&{\$80,000}&{\$80,000}&{\$80,000}&{\$80,000}&{\$80,000}&{\$80,000}&{\$80,000}&{\$80,000} \\ \hline
{Current actuals}&{\$20,000}&{\$20,000}&{\$20,000}&{\$20,000}&{\$20,000}&{\$20,000}&{\$20,000}&{\$20,000}&{\$20,000}&{\$20,000}&{\$20,000} \\ \hline
{RPF sims, V\&V}&{\$10,000}&{\$10,000}&{\$10,000}&{\$10,000}&{\$10,000}&{\$10,000}&{\$10,000}&{\$10,000}&{\$10,000}&{\$10,000}& \\ \hline
\textbf{Total}&\textbf{\$110,000}&\textbf{\$110,000}&\textbf{\$110,000}&\textbf{\$110,000}&\textbf{\$110,000}&\textbf{\$110,000}&\textbf{\$110,000}&\textbf{\$110,000}&\textbf{\$110,000}&\textbf{\$110,000}&\textbf{\$100,000} \\ \hline
\end{longtable} \normalsize



More details on the inputs are in \secref{sec:opsdetails}

\subsection{US and Chile}
While \tabref{tab:opsSummary} present the total ops cost for Rubin Observatory a fraction of this is in Chile and would potentially remain an NSF cost in operations. \tabref{tab:opsSumUSDF} presents just the US Data Facility budget  and
\tabref{tab:opsChileR} presents the Chile budget.

\tiny \begin{longtable} { |p{0.22\textwidth}  |r  |r  |r  |r  |r  |r  |r  |r  |r  |r  |r |} 
\caption{This table pulls together all the information in a high level summary for USDF operations - in this table Rome pricing(see \tabref{tab:opsRomeUSDF}). Price factors, defined in \tabref{tab:Inputs} are applied in all cases - other input values come from \tabref{tab:opsInputs}, \tabref{tab:opsStorageUSDF}.
 \label{tab:opsSumUSDF}}\\ 
\hline 
\textbf{Year  (all prices Million\$)}&\textbf{2024}&\textbf{2025}&\textbf{2026}&\textbf{2027}&\textbf{2028}&\textbf{2029}&\textbf{2030}&\textbf{2031}&\textbf{2032}&\textbf{2033} \\ \hline
{Compute (2020 pricing)}&{\$0.65}&{\$0.69}&{\$1.14}&{\$1.43}&{\$1.47}&{\$1.66}&{\$1.55}&{\$1.55}&{\$1.66}&{\$1.55} \\ \hline
{Qserv (2020 pricing)}&{\$1.62}&{\$2.42}&{\$1.86}&{\$2.40}&{\$2.74}&{\$3.60}&{\$2.10}&{\$2.18}&{\$2.78}&{\$3.12} \\ \hline
{Storage (2020 pricing)}&{\$7.92}&{\$9.73}&{\$11.22}&{\$10.94}&{\$12.62}&{\$18.97}&{\$20.83}&{\$22.33}&{\$22.05}&{\$22.47} \\ \hline
\textbf{Total (2019 pricing)}&\textbf{\$10.19}&\textbf{\$12.84}&\textbf{\$14.22}&\textbf{\$14.77}&\textbf{\$16.83}&\textbf{\$24.23}&\textbf{\$24.48}&\textbf{\$26.06}&\textbf{\$26.49}&\textbf{\$27.14} \\ \hline
{Applying price factor (CPU)}&{\$0.43}&{\$0.41}&{\$0.61}&{\$0.68}&{\$0.63}&{\$0.64}&{\$0.54}&{\$0.49}&{\$0.47}&{\$0.39} \\ \hline
{IN2P3 (50\% of compute)}&{-\$0.21}&{-\$0.20}&{-\$0.30}&{-\$0.34}&{-\$0.32}&{-\$0.32}&{-\$0.27}&{-\$0.24}&{-\$0.23}&{-\$0.20} \\ \hline
{UKDF (25\% of compute)}&{-\$0.11}&{-\$0.10}&{-\$0.15}&{-\$0.17}&{-\$0.16}&{-\$0.16}&{-\$0.14}&{-\$0.12}&{-\$0.12}&{-\$0.10} \\ \hline
{Qserv (applying factor)}&{\$1.12}&{\$1.53}&{\$1.07}&{\$1.26}&{\$1.32}&{\$1.58}&{\$0.84}&{\$0.80}&{\$0.93}&{\$0.95} \\ \hline
{Applying price factor (Storage)}&{\$5.80}&{\$6.59}&{\$7.03}&{\$6.34}&{\$6.76}&{\$9.41}&{\$9.55}&{\$9.47}&{\$8.65}&{\$8.16} \\ \hline
{Hosting Overhead SLAC
}&{0.5}&{0.5}&{0.5}&{0.5}&{0.5}&{0.7}&{0.6}&{0.6}&{0.6}&{0.6} \\ \hline
\textbf{Total budget (using price factors)}&\textbf{\$7.48}&\textbf{\$8.74}&\textbf{\$8.77}&\textbf{\$8.26}&\textbf{\$8.75}&\textbf{\$11.80}&\textbf{\$11.16}&\textbf{\$11.01}&\textbf{\$10.28}&\textbf{\$9.76} \\ \hline
\textbf{Total Operations hardware to 2033}&\textbf{\$96.01}&\textbf{million}&\textbf{to 2035}&\textbf{\$100.94}&&&&&& \\ \hline
\end{longtable} \normalsize


Note that in \tabref{tab:opsChileR} a fraction of the USDF storage is considered, big enough for Raws and a cache of products. It also does not include a Qserv. See also \tabref{tab:opsSumChile}
\tiny \begin{longtable} { |p{0.22\textwidth}  |r  |r  |r  |r  |r  |r  |r  |r  |r  |r  |r |} 
\caption{"This table pulls together all the information in a high level summary 
for Chile operations - in this table Rome pricing(see 
\tabref{tab:opsRomeChile}) is used  Price factors, defined in 
\tabref{tab:Inputs} are applied in all cases - other input values come 
from \tabref{tab:opsInputs}, \tabref{tab:opsStorageChile}. \label{tab:opsChileR}}\\ 
\hline 
\textbf{Year  (all prices Million\$)}&\textbf{2024}&\textbf{2025}&\textbf{2026}&\textbf{2027}&\textbf{2028}&\textbf{2029}&\textbf{2030}&\textbf{2031}&\textbf{2032}&\textbf{2033} \\ \hline
{Compute (2020 pricing)}&{\$0.04}&{\$0.04}&{\$0.04}&{\$0.08}&{\$0.08}&{\$0.08}&{\$0.08}&{\$0.08}&{\$0.08}&{\$0.08} \\ \hline
{Storage (2020 pricing)}&{\$1.21}&{\$1.50}&{\$1.71}&{\$1.51}&{\$1.50}&{\$2.71}&{\$3.00}&{\$3.21}&{\$3.01}&{\$3.00} \\ \hline
\textbf{Total (2020 pricing)}&\textbf{\$1.25}&\textbf{\$1.54}&\textbf{\$1.75}&\textbf{\$1.59}&\textbf{\$1.58}&\textbf{\$2.79}&\textbf{\$3.08}&\textbf{\$3.29}&\textbf{\$3.09}&\textbf{\$3.08} \\ \hline
{Applying price factor (CPU)}&{\$0.03}&{\$0.02}&{\$0.02}&{\$0.04}&{\$0.03}&{\$0.03}&{\$0.02}&{\$0.02}&{\$0.02}&{\$0.02} \\ \hline
{Applying price factor (Storage)}&{\$0.89}&{\$1.02}&{\$1.07}&{\$0.88}&{\$0.80}&{\$1.35}&{\$1.38}&{\$1.36}&{\$1.18}&{\$1.09} \\ \hline
{Overhead hosting Chile}&{\$0.15}&{\$0.13}&{\$0.08}&{\$0.05}&{\$0.05}&{\$0.14}&{\$0.03}&{\$0.03}&{\$0.02}&{\$0.03} \\ \hline
\textbf{Total budget (using price factors)}&\textbf{\$1.07}&\textbf{\$1.17}&\textbf{\$1.17}&\textbf{\$0.96}&\textbf{\$0.88}&\textbf{\$1.52}&\textbf{\$1.43}&\textbf{\$1.41}&\textbf{\$1.22}&\textbf{\$1.13} \\ \hline
\textbf{Total Chile hardware to 2033}&\textbf{\$11.97}&\textbf{million}&\textbf{to 2035}&\textbf{\$12.82}&&&&&& \\ \hline
\end{longtable} \normalsize



\section{Cost details}
The summary table (\tabref{tab:Summary})  uses Xeon pricing for compute as shown in
\tabref{tab:Xeon}.
\tiny \begin{longtable} { |p{0.22\textwidth}  |r  |r  |r  |r  |r |} 
\caption{Implementation with Intel Xeon \label{tab:Xeon}}\\ 
\hline 
\textbf{Year}&\textbf{2020}&\textbf{2021}&\textbf{2022}&\textbf{2023} \\ \hline
{Number of Xeon}&{30.01}&{0.00}&{328.68}&{1,564.11} \\ \hline
{Approximate cost}&{\$300,102.25}&{\$0.00}&{\$3,286,840.87}&{\$15,641,099.39} \\ \hline
\end{longtable} \normalsize


An alternative architecture would be Rome - SLAC have chosen this for the Ops pricing, \tabref{tab:Rome} gives the price of compute based on Rome -small and large.
Rome large are used in the operations calculations.
\tiny \begin{longtable} { |p{0.22\textwidth}  |r  |r  |r  |r  |r |} 
\caption{Implementation with AMD Rome (we have no good proce for these reallly) \label{tab:Rome}}\\ 
\hline 
\textbf{Year}&\textbf{2021}&\textbf{2022}&\textbf{2023}&\textbf{2024} \\ \hline
{number of small rome }&{49}&{0}&{75}&{388} \\ \hline
{Approximate cost of small rome }&{\$637,000.00}&{\$0.00}&{\$975,000.00}&{\$5,044,000.00} \\ \hline
{number of large rome }&{16}&{0}&{25}&{127} \\ \hline
{Approximate cost of large rome }&{\$208,000.00}&{\$0.00}&{\$325,000.00}&{\$1,651,000.00} \\ \hline
\end{longtable} \normalsize


\tabref{tab:StorageCost} gives the price of storage using all  types that we need.
This would be needed regardless of the compute chosen.
\tiny \begin{longtable} { |p{0.22\textwidth}  |r  |r  |r  |r  |r |} 
\caption{Total storage cost estimate \label{tab:StorageCost}}\\ 
\hline 
\textbf{Year}&\textbf{2020}&\textbf{2021}&\textbf{2022}&\textbf{2023} \\ \hline
{Fast Storage}&{\$4,736.84}&{\$4,736.84}&{\$10,428.00}&{\$62,568.00} \\ \hline
{Fast Storage Chile}&{\$0.00}&{\$0.00}&{\$0.00}&{\$62,568.00} \\ \hline
{Normal Storage}&{\$90,405.06}&{\$9,063.48}&{\$730,646.97}&{\$3,955,645.91} \\ \hline
{Latent Storage}&{\$42,363.20}&{\$74,135.60}&{\$543,948.46}&{\$3,177,069.33} \\ \hline
{Latent Storage Chile}&{\$0.00}&{\$0.00}&{\$0.00}&{\$3,837,516.59} \\ \hline
{High Latency Storage}&{\$45,549.15}&{\$50,354.27}&{\$165,965.39}&{\$727,912.22} \\ \hline
\textbf{Total}&\textbf{\$183,054.25}&\textbf{\$138,290.19}&\textbf{\$1,450,988.82}&\textbf{\$11,823,280.05} \\ \hline
\end{longtable} \normalsize


\tabref{tab:overheadCost} gives the annual cost of hosting compute at NCSA for construction.
This includes purchasing racks to house
new nodes etc.
\tiny \begin{longtable} { |p{0.22\textwidth}  |r  |r  |r  |r  |r |} 
\caption{Overheads(NCSA) per year based on number of cores in \tabref{tab:Inputs} and costs in \tabref{tab:overheads} assuming Xeon density from \tabref{tab:Machines}.  \label{tab:overheadCost}}\\ 
\hline 
\textbf{Year}&\textbf{2020}&\textbf{2021}&\textbf{2022}&\textbf{2023} \\ \hline
\textbf{Total Incremental cores (USA)}&\textbf{1,836}&\textbf{0}&\textbf{4,026}&\textbf{7,521} \\ \hline
\textbf{Total owned cores (USA)}&\textbf{3,528}&\textbf{3,528}&\textbf{7,554}&\textbf{15,075} \\ \hline
\textbf{Total owned nodes}&\textbf{111}&\textbf{111}&\textbf{251}&\textbf{567} \\ \hline
{Cost for hosting nodes}&{\$62,802}&{\$62,802}&{\$142,012}&{\$320,801} \\ \hline
\textbf{Total new nodes}&\textbf{58}&\textbf{0}&\textbf{140}&\textbf{317} \\ \hline
\textbf{Total new racks}&\textbf{2}&\textbf{0}&\textbf{4}&\textbf{9} \\ \hline
{Rack install cost }&{\$48,000.00}&{\$0.00}&{\$96,000.00}&{\$216,000.00} \\ \hline
\textbf{Total Overhead (NCSA)}&\textbf{\$110,802.27}&\textbf{\$62,802.27}&\textbf{\$238,012.35}&\textbf{\$536,800.80} \\ \hline
\end{longtable} \normalsize


\subsection{Ops Cost details}\label{sec:opsdetails}
\tabref{tab:opsXeon} gives the price of compute based on Xeons.

\tabref{tab:opsStorageCost} gives the price of storage using all  types that we need.
This would be needed regardless of the compute chosen.
\begin{landscape}
\tiny \begin{longtable} { |p{0.22\textwidth}  |r  |r  |r  |r  |r  |r  |r  |r  |r  |r  |r |} 
\caption{Implementation with Intel Xeon \label{tab:opsXeon}}\\ 
\hline 
\textbf{Year}&\textbf{2023}&\textbf{2024}&\textbf{2025}&\textbf{2026}&\textbf{2027}&\textbf{2028}&\textbf{2029}&\textbf{2030}&\textbf{2031}&\textbf{2032} \\ \hline
{Number of Xeon}&{282}&{298}&{470}&{618}&{634}&{694}&{672}&{672}&{694}&{672} \\ \hline
{Approximate cost (2019 Mdollars)}&{\$2.82}&{\$2.98}&{\$4.70}&{\$6.18}&{\$6.34}&{\$6.94}&{\$6.72}&{\$6.72}&{\$6.94}&{\$6.72} \\ \hline
\end{longtable} \normalsize

\tiny \begin{longtable} { |p{0.22\textwidth}  |r  |r  |r  |r  |r  |r  |r  |r  |r  |r  |r |} 
\caption{Total storage cost estimate for operations of Rubin Observatory USDF and CHile \label{tab:opsStorageCost}}\\ 
\hline 
\textbf{Year (all in M\$)}&\textbf{2024}&\textbf{2025}&\textbf{2026}&\textbf{2027}&\textbf{2028}&\textbf{2029}&\textbf{2030}&\textbf{2031}&\textbf{2032}&\textbf{2033} \\ \hline
{Fast Storage}&{\$0.20}&{\$0.16}&{\$0.19}&{\$0.13}&{\$0.12}&{\$0.31}&{\$0.27}&{\$0.30}&{\$0.23}&{\$0.22} \\ \hline
{Normal Storage}&{\$3.96}&{\$3.86}&{\$4.20}&{\$4.19}&{\$4.89}&{\$8.13}&{\$8.05}&{\$8.41}&{\$8.42}&{\$8.41} \\ \hline
{Latent Storage}&{\$5.36}&{\$6.44}&{\$7.00}&{\$5.39}&{\$5.94}&{\$10.75}&{\$11.83}&{\$12.39}&{\$10.78}&{\$10.78} \\ \hline
{High Latency Storage}&{\$0.73}&{\$1.13}&{\$1.56}&{\$2.00}&{\$2.43}&{\$2.86}&{\$3.28}&{\$3.72}&{\$4.15}&{\$4.58} \\ \hline
\textbf{Total (M\$)}&\textbf{\$10.24}&\textbf{\$11.58}&\textbf{\$12.95}&\textbf{\$11.70}&\textbf{\$13.38}&\textbf{\$22.04}&\textbf{\$23.44}&\textbf{\$24.82}&\textbf{\$23.58}&\textbf{\$23.99} \\ \hline
\end{longtable} \normalsize

\end{landscape}

\tabref{tab:overheadCost} gives the annual cost of hosting compute. This includes purchasing racks to house
new nodes etc.
\begin{landscape}
\tiny \begin{longtable} { |p{0.22\textwidth}  |r  |r  |r  |r  |r  |r  |r  |r  |r  |r  |r |} 
\caption{Overheads(NCSA) per year based on number of cores in \tabref{tab:opsInputs} and costs in \tabref{tab:overheads} assuming Xeon density from \tabref{tab:Machines}.  \label{tab:opsOverheadCost}}\\ 
\hline 
\textbf{Year}&\textbf{2023}&\textbf{2024}&\textbf{2025}&\textbf{2026}&\textbf{2027}&\textbf{2028}&\textbf{2029}&\textbf{2030}&\textbf{2031}&\textbf{2032} \\ \hline
\textbf{Total Incremental cores (USA)}&\textbf{7,521}&\textbf{7,958}&\textbf{8,979}&\textbf{8,979}&\textbf{8,979}&\textbf{8,979}&\textbf{8,979}&\textbf{8,979}&\textbf{8,979}&\textbf{8,979} \\ \hline
\textbf{Total owned cores (USA)}&\textbf{15,075}&\textbf{23,033}&\textbf{32,012}&\textbf{40,990}&\textbf{49,969}&\textbf{58,947}&\textbf{67,926}&\textbf{76,904}&\textbf{85,883}&\textbf{94,861} \\ \hline
\textbf{Total owned nodes}&\textbf{567}&\textbf{936}&\textbf{1,310}&\textbf{1,629}&\textbf{1,926}&\textbf{2,294}&\textbf{2,559}&\textbf{2,812}&\textbf{3,051}&\textbf{3,383} \\ \hline
{Cost for hosting nodes}&{\$320,801}&{\$529,576}&{\$741,180}&{\$921,666}&{\$1,089,704}&{\$1,297,914}&{\$1,447,847}&{\$1,590,991}&{\$1,726,214}&{\$1,914,055} \\ \hline
\textbf{Total new nodes}&\textbf{317}&\textbf{370}&\textbf{374}&\textbf{401}&\textbf{418}&\textbf{461}&\textbf{386}&\textbf{390}&\textbf{420}&\textbf{437} \\ \hline
\textbf{Total new racks}&\textbf{9}&\textbf{11}&\textbf{11}&\textbf{12}&\textbf{12}&\textbf{13}&\textbf{11}&\textbf{11}&\textbf{12}&\textbf{13} \\ \hline
{Rack install cost }&{\$216,000.00}&{\$264,000.00}&{\$264,000.00}&{\$288,000.00}&{\$288,000.00}&{\$312,000.00}&{\$264,000.00}&{\$264,000.00}&{\$288,000.00}&{\$312,000.00} \\ \hline
\textbf{Total Ops Overhead (NCSA)}&\textbf{\$536,800.80}&\textbf{\$793,575.92}&\textbf{\$1,005,179.97}&\textbf{\$1,209,665.78}&\textbf{\$1,377,704.29}&\textbf{\$1,609,913.63}&\textbf{\$1,711,846.98}&\textbf{\$1,854,990.90}&\textbf{\$2,014,213.81}&\textbf{\$2,226,054.84} \\ \hline
\end{longtable} \normalsize

\end{landscape}
\tiny \begin{longtable} { |p{0.22\textwidth}  |r  |r  |r  |r  |r  |r  |r  |r  |r  |r  |r |} 
\caption{Various inputs for deriving costs in operations - these drive the costs in \tabref{tab:opsSummary}. This is based on \tabref{tab:opsXeon}, \tabref{tab:opsStorageCost}  \label{tab:opsInputs}}\\ 
\hline 
\textbf{Year}&\textbf{2024}&\textbf{2025}&\textbf{2026}&\textbf{2027}&\textbf{2028}&\textbf{2029}&\textbf{2030}&\textbf{2031}&\textbf{2032}&\textbf{2033} \\ \hline
{Core-hours Needed Total (DRP)}&{4.5E+07}&{8.2E+07}&{1.2E+08}&{1.6E+08}&{2.0E+08}&{2.5E+08}&{2.9E+08}&{3.3E+08}&{3.7E+08}&{4.1E+08} \\ \hline
{Core-hours Annual Increase}&{3.40E+07}&{3.6E+07}&{4.1E+07}&{4.1E+07}&{4.1E+07}&{4.1E+07}&{4.1E+07}&{4.1E+07}&{4.1E+07}&{4.1E+07} \\ \hline
{Time to Process days}&{200}&{200}&{200}&{200}&{200}&{200}&{200}&{200}&{200}&{200} \\ \hline
{Time to Process hours}&{4,800}&{4,800}&{4,800}&{4,800}&{4,800}&{4,800}&{4,800}&{4,800}&{4,800}&{4,800} \\ \hline
{Cores (DRP) Annual increase}&{7,093}&{7,594}&{8,512}&{8,512}&{8,512}&{8,512}&{8,512}&{8,512}&{8,512}&{8,512} \\ \hline
{Cores (DRP) Annual refresh}&{}&{}&{2,837}&{7,093}&{7,594}&{8,512}&{8,512}&{8,512}&{8,512}&{8,512} \\ \hline
{Cores (DRP) Annual purchase}&{7,093}&{7,594}&{11,349}&{15,605}&{16,106}&{17,024}&{17,024}&{17,024}&{17,024}&{17,024} \\ \hline
{Cores (Alerts)}&{1,188}&{1,188}&{1,188}&{1,188}&{1,188}&{1,188}&{1,188}&{1,188}&{1,188}&{1,188} \\ \hline
{Cores (Alerts) Annual refresh}&{}&{}&{1,188}&{}&{}&{1,188}&{}&{}&{1,188}& \\ \hline
{Cores (US DAC/ Staff)}&{568}&{933}&{1,399}&{1,866}&{2,332}&{2,798}&{3,265}&{3,731}&{4,198}&{4,664} \\ \hline
{Cores (US DAC/ Staff) Annual increase}&{428}&{364}&{466}&{466}&{466}&{466}&{466}&{466}&{466}&{466} \\ \hline
{Cores (US DAC/ Staff) Annual refresh}&{}&{}&{141}&{428}&{364}&{466}&{466}&{466}&{466}&{466} \\ \hline
{Cores (US DAC/ Staff) Annual purchase}&{428}&{364}&{607}&{894}&{831}&{933}&{933}&{933}&{933}&{933} \\ \hline
{Cores (Chilean DAC)}&{103}&{187}&{280}&{373}&{466}&{560}&{653}&{746}&{840}&{933} \\ \hline
{Cores (Chilean DAC) Annual increase}&{103}&{83}&{93}&{93}&{93}&{93}&{93}&{93}&{93}&{93} \\ \hline
{Cores (Chilean DAC) Annual refresh}&{}&{}&{0}&{103}&{83}&{93}&{93}&{93}&{93}&{93} \\ \hline
{Cores (Chilean DAC) Annual purchase}&{103}&{83}&{93}&{197}&{177}&{187}&{187}&{187}&{187}&{187} \\ \hline
{Qserv nodes (US DAC/ Staff)}&{95}&{216}&{309}&{348}&{364}&{451}&{436}&{408}&{367}&{418} \\ \hline
{Qserv nodes (US DAC/ Staff) Annual Increase}&{81}&{121}&{93}&{120}&{137}&{180}&{105}&{109}&{139}&{156} \\ \hline
{Qserv nodes (Chilean DAC)}&{95}&{216}&{309}&{348}&{364}&{451}&{436}&{408}&{367}&{418} \\ \hline
{Qserv nodes (Chilean DAC) Annual Increase}&{95}&{121}&{93}&{134}&{137}&{180}&{119}&{109}&{139}&{170} \\ \hline
\textbf{Total Cores Annual Increase}&\textbf{7,624}&\textbf{8,042}&\textbf{13,238}&\textbf{16,696}&\textbf{17,113}&\textbf{19,332}&\textbf{18,144}&\textbf{18,144}&\textbf{19,332}&\textbf{18,144} \\ \hline
{Fast Storage (TB)}&{206}&{371}&{586}&{667}&{735}&{798}&{859}&{918}&{974}&{1029} \\ \hline
{Annual Increase (Fast)}&{156}&{164}&{215}&{81}&{68}&{63}&{60}&{59}&{57}&{55} \\ \hline
{Annual Refresh (Fast)}&{}&{}&{}&{}&{26}&{156}&{164}&{215}&{81}&{68} \\ \hline
{Annual Purchase (Fast)}&{156}&{164}&{215}&{81}&{94}&{220}&{225}&{275}&{138}&{123} \\ \hline
{Normal Storage (TB)}&{38,983}&{67982}&{99544}&{131031}&{162327}&{193733}&{225294}&{256997}&{288794}&{320737} \\ \hline
{Annual Increase (Normal)}&{29,742}&{28999}&{31563}&{31487}&{31296}&{31406}&{31560}&{31703}&{31797}&{31943} \\ \hline
{Annual Refresh (Normal)}&{}&{}&{}&{}&{5,494}&{29,742}&{28,999}&{31,563}&{31,487}&{31,296} \\ \hline
{Annual Purchase (Normal)}&{29,742}&{28,999}&{31,563}&{31,487}&{36,790}&{61,148}&{60,559}&{63,266}&{63,284}&{63,239} \\ \hline
{Latent Storage  (TB)}&{28,854}&{64,086}&{104,491}&{139,969}&{175,447}&{210,925}&{246,403}&{281,881}&{317,359}&{352,837} \\ \hline
{Annual Increase (Latent)}&{23,888}&{35,232}&{40,405}&{35,478}&{35,478}&{35,478}&{35,478}&{35,478}&{35,478}&{35,478} \\ \hline
{Annual Refresh (Latent)}&{}&{}&{}&{}&{4,090}&{23,888}&{35,232}&{40,405}&{35,478}&{35,478} \\ \hline
{Annual Purchase (Latent)}&{23,888}&{35,232}&{40,405}&{35,478}&{39,568}&{59,366}&{70,710}&{75,884}&{70,956}&{70,956} \\ \hline
{High Latency (TB)}&{63,245}&{135,135}&{234,943}&{362,508}&{517,497}&{699,929}&{909,829}&{1,147,221}&{1,412,122}&{1,704,554} \\ \hline
{Annual Increase (High Latency)}&{46,512}&{71,890}&{99,809}&{127,565}&{154,989}&{182,432}&{209,900}&{237,392}&{264,900}&{292,432} \\ \hline
{Chilean DAC Fast Storage (TB)}&{156}&{347}&{562}&{643}&{711}&{774}&{835}&{894}&{951}&{1,006} \\ \hline
{Annual Increase (Fast Chilean DAC)}&{156}&{190}&{215}&{81}&{68}&{63}&{60}&{59}&{57}&{55} \\ \hline
{Annual Refresh (Fast Chilean DAC)}&{}&{}&{}&{}&{}&{156}&{190}&{215}&{81}&{68} \\ \hline
{Annual Purchase (Fast Chilean DAC)}&{156}&{190}&{215}&{81}&{68}&{220}&{251}&{275}&{138}&{123} \\ \hline
{Chilean DAC Latent Storage (TB)}&{28,854}&{64,086}&{104,491}&{139,969}&{175,447}&{210,925}&{246,403}&{281,881}&{317,359}&{352,837} \\ \hline
{Annual Increase (Latent Chilean DAC)}&{28,854}&{35,232}&{40,405}&{35,478}&{35,478}&{35,478}&{35,478}&{35,478}&{35,478}&{35,478} \\ \hline
{Annual Refresh (Latent Chilean DAC)}&{}&{}&{}&{}&{}&{28,854}&{35,232}&{40,405}&{35,478}&{35,478} \\ \hline
{Annual Purchase (Latent Chilean DAC)}&{28,854}&{35,232}&{40,405}&{35,478}&{35,478}&{64,332}&{70,710}&{75,884}&{70,956}&{70,956} \\ \hline
\end{longtable} \normalsize





\section{ Models}\label{sec:model}
\subsection{Sizing model}\label{sec:sizemodel}

An exhaustive and detailed mode is provided in  \citedsp{LDM-138,LDM-144} - here we concentrate on the needs for the
final years of construction. We explore the compute and storage needed to get us through commissioning and suggest a 2023 purchase for DR1,2 processing which could be pushed to operations.

\tabref{tab:Inputs} gives the annual requirements for the next few years.

\tiny \begin{longtable} { |p{0.22\textwidth}  |r  |r  |r  |r  |r  |r |} 
\caption{Various inputs for deriving costs - 2019 represents currentl holdings. \label{tab:Inputs}}\\ 
\hline 
\textbf{Year}&\textbf{2019}&\textbf{2020}&\textbf{2021}&\textbf{2022}&\textbf{2023} \\ \hline
{Core-hours Needed Total (DRP)}&{}&{4.41E+06}&{4.41E+06}&{1.12E+07}&{4.53E+07} \\ \hline
{Annual Increase}&{}&{4.41E+06}&{0.00E+00}&{6.81E+06}&{3.40E+07} \\ \hline
{Time to Process days}&{}&{100.0}&{100.0}&{100.0}&{200} \\ \hline
{Time to Process hours}&{}&{2,400}&{2,400}&{2,400}&{4,800} \\ \hline
{Instantaneous cores (DRP) Annual increase}&{1152}&{1,836}&{0}&{2,837}&{7,093} \\ \hline
{Instantaneous cores (Alerts)}&{}&{0}&{0}&{594}&{594} \\ \hline
{Cores (Alerts) Annual increase}&{}&{0}&{0}&{594}&{0} \\ \hline
{Instantaneous cores (US DAC/ Staff)}&{538}&{538}&{538}&{141}&{568} \\ \hline
{Cores (US DAC/ Staff) Annual increase}&{}&{0}&{0}&{0}&{428} \\ \hline
{Instantaneous cores (Chilean DAC)}&{}&{0}&{0}&{26}&{103} \\ \hline
{Cores (Chilean DAC) Annual increase}&{}&{0}&{0}&{26}&{78} \\ \hline
{Qserv nodes (US DAC/ Staff)}&{}&{}&{}&{14}&{95} \\ \hline
{Qserv nodes (US DAC/ Staff) Annual Increase}&{}&{}&{}&{14}&{81} \\ \hline
{Qserv nodes (Chilean DAC)}&{}&{}&{}&{14}&{95} \\ \hline
{Qserv nodes (Chilean DAC) Annual Increase}&{}&{}&{}&{14}&{81} \\ \hline
\textbf{Total Annual Increase}&\textbf{}&\textbf{1,836}&\textbf{0}&\textbf{3,457}&\textbf{7,599} \\ \hline
{Fast Storage (TB)}&{}&{12}&{24}&{50}&{206} \\ \hline
{Annual Increase (Fast)}&{}&{12}&{12}&{26}&{156} \\ \hline
{Normal Storage (TB)}&{3000}&{3391}&{3459}&{9317}&{41786} \\ \hline
{Annual Increase (Normal)}&{}&{391}&{68}&{5857}&{32469} \\ \hline
{Latent Storage  (TB)}&{}&{319}&{876}&{4057}&{20217} \\ \hline
{Annual Increase (Latent)}&{}&{319}&{557}&{3181}&{16160} \\ \hline
{High Latency (TB)}&{}&{3710}&{7727}&{18333}&{64935} \\ \hline
{Annual Increase (High Latency)}&{}&{3710}&{4017}&{10607}&{46602} \\ \hline
{Chilean DAC Fast Storage (TB)}&{}&{}&{}&{}&{156} \\ \hline
{Annual Increase (Fast Chilean DAC)}&{}&{}&{}&{}&{156} \\ \hline
{Chilean DAC Latent Storage (TB)}&{}&{}&{}&{}&{20217} \\ \hline
{Annual Increase (Latent Chilean DAC)}&{}&{}&{}&{}&{20217} \\ \hline
{Annual price decrease CPU}&{}&{10\%}&{}&{}&{} \\ \hline
{Annual price decrease Storage}&{}&{5\%}&{}&{}&{} \\ \hline
{Annual price decrease Qserv}&{}&{8\%}&{}&{}&{} \\ \hline
\end{longtable} \normalsize


\subsection{Compute and storage }\label{sec:csmodel}
We which to base our budget on reasonable well know machines for which we have well know prices.
\tabref{tab:Machines} gives an outline of a few standard machines we use and a price. This table also gives a FLOP estimate
for those machines.
\tabref{tab:Storage} gives costs for different types of storage - we will require various latency for different tasks
and those have varying costs.
These tables are used as look ups for the cost models in \secref{sec:cost}

\tiny \begin{longtable} { |p{0.22\textwidth}  |r  |r |} 
\caption{Storage types and costs used as inputs used for calculations \label{tab:Storage}}\\ 
\hline 
{Storage type }&{cost} \\ \hline
{fast -- NVME (50GB/ s each) Cost for TB  }&{\$1,000.00} \\ \hline
{normal - SATA GPFS file systems/ TB  }&{\$135.00} \\ \hline
{latency -- slower but on disk }&{\$100.00} \\ \hline
{high latency -- very slow -- on tape }&{\$64.00} \\ \hline
{}&{} \\ \hline
\end{longtable} \normalsize


In \tabref{tab:Storage} we should consider for NVMe for each TB with file system servers two DDN NVMe box with GPFS servers.
The price is based on the TOP performer with best price .
The Normal price is for each TB with file system disks and servers locally attached to production resources.

In the latency and high latency prices are only at NCSA: for each TB with file systems and all people/services.
The complete service not usually attached.   S3 bucket type.
Can be mounted if needed but not for production worthy speeds.
The complete service with data flowing to tape using policies.

\tiny \begin{longtable} { |p{0.22\textwidth}  |r  |r  |r  |r  |r  |r |} 
\caption{Machine types and costs used as inputs used for calculations \label{tab:Machines}}\\ 
\hline 
{Type of machine }&{Cores}&{Memory(GB)}&{GFLOP/ s}&{Cost}&{purpose/ use } \\ \hline
{xeon }&{32}&{192}&{12.672}&{\$10,000.00}&{current K8 node } \\ \hline
{qserv }&{12}&{128}&{4.104}&{\$20,000.00}&{current qserv node } \\ \hline
{small rome  }&{64}&{256}&{374.4}&{\$15,000.00}&{https:/ / www.microway.com/ product/ navion-1u-amd-epyc-gpu-server/ } \\ \hline
{large rome }&{128}&{512}&{748.8}&{\$25,000.00}& \\ \hline
{}&{}&{}&{}&{}&{} \\ \hline
{}&{}&{}&{}&{}&{} \\ \hline
\end{longtable} \normalsize


There is also an associated running cost for machines included in the total cost of ownership.
These overheads are listed in \tabref{tab:overheads}.

\tiny \begin{longtable} { |p{0.22\textwidth}  |r  |r |} 
\caption{Overhead costs per rack \label{tab:overheads}}\\ 
\hline 
\textbf{Item}&\textbf{Number/ Cost} \\ \hline
{Compute nodes in a rack }&{36} \\ \hline
{Rack initial cost has power, networking switches, networking cables, ready for machine installation-- switches last 5 years.  Will need to refresh, but rack should last entire project.  }&{\$24,000.00} \\ \hline
{ ** need to add annually: floor space for rack for 1 years.   need to renew after new nodes are racked/ stacked }&{\$300} \\ \hline
{** Need to add annually: power for 1 node for 1 yr - kw * rate * hours/ year * }&{\$348} \\ \hline
{** need to add annually: cooling for 1 node for 5 years  kw* chillded water per MTBU* hours/ year *  1KW in (MTBU) }&{\$210} \\ \hline
{** Need to add annually: maintenance for node s -- can't purchase more than what the contract has in time left.  could be included in the price of the machine, and might not be added in here.  }&{\$1,500} \\ \hline
{Cost for each machine for 1 year in a rack.   }&{\$566} \\ \hline
{**** need to add in at an annual basis.  software maintenance (oracle and other software not associated with specific node annually)  Oracle license, VM licensing.  }&{\$35,000} \\ \hline
{Compute node lifetime (years)}&{3} \\ \hline
{Storage lifetime (years)}&{5} \\ \hline
\end{longtable} \normalsize



