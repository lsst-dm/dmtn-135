\section{Introduction}



\section{ Models}\label{sec:model}
\subsection{Sizing model}\label{sec:sizemodel}

An exhaustive and detailed mode is provided in  \citedsp{LDM-138,LDM-144} - here we concentrate on the needs for the
final years of construction. We explore the compute and storage needed to get us through commissioning and suggest a 2023 purchase for DR1,2 processing which could be pushed to operations.

\tabref{tab:Inputs} gives the annual requirements for the next few years.

\tiny \begin{longtable} { |p{0.22\textwidth}  |r  |r  |r  |r  |r  |r |} 
\caption{Various inputs for deriving costs - 2019 represents currentl holdings. \label{tab:Inputs}}\\ 
\hline 
\textbf{Year}&\textbf{2019}&\textbf{2020}&\textbf{2021}&\textbf{2022}&\textbf{2023} \\ \hline
{Core-hours Needed Total (DRP)}&{}&{4.41E+06}&{4.41E+06}&{1.12E+07}&{4.53E+07} \\ \hline
{Annual Increase}&{}&{4.41E+06}&{0.00E+00}&{6.81E+06}&{3.40E+07} \\ \hline
{Time to Process days}&{}&{100.0}&{100.0}&{100.0}&{200} \\ \hline
{Time to Process hours}&{}&{2,400}&{2,400}&{2,400}&{4,800} \\ \hline
{Instantaneous cores (DRP) Annual increase}&{1152}&{1,836}&{0}&{2,837}&{7,093} \\ \hline
{Instantaneous cores (Alerts)}&{}&{0}&{0}&{594}&{594} \\ \hline
{Cores (Alerts) Annual increase}&{}&{0}&{0}&{594}&{0} \\ \hline
{Instantaneous cores (US DAC/ Staff)}&{538}&{538}&{538}&{141}&{568} \\ \hline
{Cores (US DAC/ Staff) Annual increase}&{}&{0}&{0}&{0}&{428} \\ \hline
{Instantaneous cores (Chilean DAC)}&{}&{0}&{0}&{26}&{103} \\ \hline
{Cores (Chilean DAC) Annual increase}&{}&{0}&{0}&{26}&{78} \\ \hline
{Qserv nodes (US DAC/ Staff)}&{}&{}&{}&{14}&{95} \\ \hline
{Qserv nodes (US DAC/ Staff) Annual Increase}&{}&{}&{}&{14}&{81} \\ \hline
{Qserv nodes (Chilean DAC)}&{}&{}&{}&{14}&{95} \\ \hline
{Qserv nodes (Chilean DAC) Annual Increase}&{}&{}&{}&{14}&{81} \\ \hline
\textbf{Total Annual Increase}&\textbf{}&\textbf{1,836}&\textbf{0}&\textbf{3,457}&\textbf{7,599} \\ \hline
{Fast Storage (TB)}&{}&{12}&{24}&{50}&{206} \\ \hline
{Annual Increase (Fast)}&{}&{12}&{12}&{26}&{156} \\ \hline
{Normal Storage (TB)}&{3000}&{3391}&{3459}&{9317}&{41786} \\ \hline
{Annual Increase (Normal)}&{}&{391}&{68}&{5857}&{32469} \\ \hline
{Latent Storage  (TB)}&{}&{319}&{876}&{4057}&{20217} \\ \hline
{Annual Increase (Latent)}&{}&{319}&{557}&{3181}&{16160} \\ \hline
{High Latency (TB)}&{}&{3710}&{7727}&{18333}&{64935} \\ \hline
{Annual Increase (High Latency)}&{}&{3710}&{4017}&{10607}&{46602} \\ \hline
{Chilean DAC Fast Storage (TB)}&{}&{}&{}&{}&{156} \\ \hline
{Annual Increase (Fast Chilean DAC)}&{}&{}&{}&{}&{156} \\ \hline
{Chilean DAC Latent Storage (TB)}&{}&{}&{}&{}&{20217} \\ \hline
{Annual Increase (Latent Chilean DAC)}&{}&{}&{}&{}&{20217} \\ \hline
{Annual price decrease CPU}&{}&{10\%}&{}&{}&{} \\ \hline
{Annual price decrease Storage}&{}&{5\%}&{}&{}&{} \\ \hline
{Annual price decrease Qserv}&{}&{8\%}&{}&{}&{} \\ \hline
\end{longtable} \normalsize


\subsection{Compute and storage }\label{sec:csmodel}
We which to base our budget on reasonable well know machines for which we have well know prices.
\tabref{tab:Machines} gives an outline of a few standard machines we use and a price. This table also gives a FLOP estimate
for those machines.
\tabref{tab:Storage} gives costs for different types of storage - we will require various latency for different tasks
and those have varying costs.
These tables are used as look ups for the cost models in \secref{sec:cost}

\tiny \begin{longtable} { |p{0.22\textwidth}  |r  |r |} 
\caption{Storage types and costs used as inputs used for calculations \label{tab:Storage}}\\ 
\hline 
{Storage type }&{cost} \\ \hline
{fast -- NVME (50GB/ s each) Cost for TB  }&{\$1,000.00} \\ \hline
{normal - SATA GPFS file systems/ TB  }&{\$135.00} \\ \hline
{latency -- slower but on disk }&{\$100.00} \\ \hline
{high latency -- very slow -- on tape }&{\$64.00} \\ \hline
{}&{} \\ \hline
\end{longtable} \normalsize


In \tabref{tab:storage} we should consider for NVME for each TB with file system servers two DDN NVME box with GPFS servers.
The price is based on the TOP performer with best price .
The Normal price is for each TB with file system disks and servers locally attached to production resources.

In the latency and high latency prices are only at NCSA: for each TB with file systems and all people/services.
The complete service not usually attached.   S3 bucket type.
Can be mounted if needed but not for production worthy speeds.
The complete service with data flowing to tape using policies.

\tiny \begin{longtable} { |p{0.22\textwidth}  |r  |r  |r  |r  |r  |r |} 
\caption{Machine types and costs used as inputs used for calculations \label{tab:Machines}}\\ 
\hline 
{Type of machine }&{Cores}&{Memory(GB)}&{GFLOP/ s}&{Cost}&{purpose/ use } \\ \hline
{xeon }&{32}&{192}&{12.672}&{\$10,000.00}&{current K8 node } \\ \hline
{qserv }&{12}&{128}&{4.104}&{\$20,000.00}&{current qserv node } \\ \hline
{small rome  }&{64}&{256}&{374.4}&{\$15,000.00}&{https:/ / www.microway.com/ product/ navion-1u-amd-epyc-gpu-server/ } \\ \hline
{large rome }&{128}&{512}&{748.8}&{\$25,000.00}& \\ \hline
{}&{}&{}&{}&{}&{} \\ \hline
{}&{}&{}&{}&{}&{} \\ \hline
\end{longtable} \normalsize



\section{Proposed budget}\label{sec:cost}
Based on the needs in \tabref{tab:Inputs} and the costs in \tabref{tab:Storage} and \tabref {tab:Machines}
the following budgets can be considered.

\subsection{Buy Xeon} \label{sec:xeon}

\tiny \begin{longtable} { |p{0.22\textwidth}  |r  |r  |r  |r  |r |} 
\caption{Implementation with Intel Xeon \label{tab:Xeon}}\\ 
\hline 
\textbf{Year}&\textbf{2020}&\textbf{2021}&\textbf{2022}&\textbf{2023} \\ \hline
{Number of Xeon}&{30.01}&{0.00}&{328.68}&{1,564.11} \\ \hline
{Approximate cost}&{\$300,102.25}&{\$0.00}&{\$3,286,840.87}&{\$15,641,099.39} \\ \hline
\end{longtable} \normalsize

