% Tables from https://docs.google.com/spreadsheets/d/1DiFTjsC4dP8XyOV7-uF0zwkl0r0jMuW9U9uELejpmn8/edit#gid=0
\section{Introduction}
The sizing and cost model fro DM has not been revised since 2014. This document presents
a somewhat simplified sizing model in \secref{sec:sizemodel} based in detailed sizing presented in \secref{sec:sizeinputs}.
\secref{sec:cost} presents a very high level budget summary for
DM hardware.
Now we begin to consider operations one possibility would be to consider the DR1 hardware
as an operations cost and only consider the commissioning hardware a construction cost.
This is laid out in \secref{sec:scope}.

\section{Proposed budget}\label{sec:cost}
Based on the needs in \tabref{tab:Inputs} and the costs in \tabref{tab:Storage} and \tabref {tab:Machines}
the following budgets can be considered.

A high level bottom line is given in \tabref{tab:Summary}.
Since Xeon is more expensive that is the number used for the budget calculation, should we get Rome and it really performs we may save a little.
The remainder of the document is all the details that went into that.

\tiny \begin{longtable} { |p{0.22\textwidth}  |r  |r  |r  |r  |r |} 
\caption{This table pulls together all the information in a high level summary - in this table Xeon pricing is used since that is the more expensive but better known option. Price factors, defined in \tabref{tab:Inputs} are applied post 2020.
 \label{tab:Summary}}\\ 
\hline 
\textbf{Year}&\textbf{2020}&\textbf{2021}&\textbf{2022}&\textbf{2023} \\ \hline
{Compute (2019 pricing)}&{\$690,000}&{\$0}&{\$1,510,000}&{\$2,820,000} \\ \hline
{Storage (2019 pricing)}&{\$190,702}&{\$126,563}&{\$1,216,867}&{\$7,864,125} \\ \hline
{Qserv (2019 pricing)}&{}&{}&{\$560,000}&{\$3,240,000} \\ \hline
\textbf{Total (2019 pricing)}&\textbf{\$880,702}&\textbf{\$126,563}&\textbf{\$3,286,867}&\textbf{\$13,924,125} \\ \hline
{Compute (applying price factor)}&{\$621,000}&{\$0}&{\$1,057,000}&{\$1,692,000} \\ \hline
{IN2P3 (50\% of compute in ops)}&{}&{}&{}&{-\$846,000} \\ \hline
{Storage (applying price factor)}&{\$181,167}&{\$113,907}&{\$1,034,337}&{\$6,291,300} \\ \hline
{Qserv (applying price factor)}&{}&{}&{\$434,000}&{\$2,268,000} \\ \hline
{Hosting cost NCSA
}&{\$110,802}&{\$62,802}&{\$238,012}&{\$536,801} \\ \hline
\textbf{Total budget (using price factors)}&\textbf{\$912,969}&\textbf{\$176,709}&\textbf{\$2,763,350}&\textbf{\$9,942,100} \\ \hline
\end{longtable} \normalsize


In \tabref{tab:Summary} we should note that IN2P3 do 50\% of processing so we reduce the processing cost by half. This does not
reduce the storage cost. We have applied a modest cost reduction assuming that processors  and disks get a little cheaper - that
percentage is given in \tabref{tab:Inputs}.

\section{Potential scope option} \label{sec:scope}
In the 2019 JSR we discussed the possibility of purchasing DR1 hardware as part of operations
rather than DM construction. \tabref{tab:Scope}
defines what this would be worth using the cost/sizing model in this document.

\tiny \begin{longtable} { |p{0.22\textwidth}  |r  |r |} 
\caption{Considering a scope option of delaying the purchase of LOY1 processing hardware and only purchasing what is needed for commissioning we would only purchase up to and including 2022 hardware of \tabref{tab:Summary}. If we consider that amount and the current remaining construction budget for hardware the potential worth of such a scope option is given in this table. \label{tab:Scope}}\\ 
\hline 
\textbf{Budget for commissioning (to 2022)}&\textbf{\$4,024,778} \\ \hline
{DM construciton budet remaining}&{\$14,000,000} \\ \hline
\textbf{Total potential to delay to ops }&\textbf{\$9,975,222} \\ \hline
\end{longtable} \normalsize


Should we do this some contingency for extra hardware must be kept in DM construction as well as some manpower
budget to aid with the transition to operations.

\section{Cost details}
\subsection{Buy Xeon for compute} \label{sec:xeon}
\tabref{tab:Xeon} gives the price of compute based on Xeons.
\tiny \begin{longtable} { |p{0.22\textwidth}  |r  |r  |r  |r  |r |} 
\caption{Implementation with Intel Xeon \label{tab:Xeon}}\\ 
\hline 
\textbf{Year}&\textbf{2020}&\textbf{2021}&\textbf{2022}&\textbf{2023} \\ \hline
{Number of Xeon}&{30.01}&{0.00}&{328.68}&{1,564.11} \\ \hline
{Approximate cost}&{\$300,102.25}&{\$0.00}&{\$3,286,840.87}&{\$15,641,099.39} \\ \hline
\end{longtable} \normalsize


\subsection{Buy Rome for compute} \label{sec:rome}
\tabref{tab:Rome} gives the price of compute based on Rome -small and large.
\tiny \begin{longtable} { |p{0.22\textwidth}  |r  |r  |r  |r  |r |} 
\caption{Implementation with AMD Rome (we have no good proce for these reallly) \label{tab:Rome}}\\ 
\hline 
\textbf{Year}&\textbf{2021}&\textbf{2022}&\textbf{2023}&\textbf{2024} \\ \hline
{number of small rome }&{49}&{0}&{75}&{388} \\ \hline
{Approximate cost of small rome }&{\$637,000.00}&{\$0.00}&{\$975,000.00}&{\$5,044,000.00} \\ \hline
{number of large rome }&{16}&{0}&{25}&{127} \\ \hline
{Approximate cost of large rome }&{\$208,000.00}&{\$0.00}&{\$325,000.00}&{\$1,651,000.00} \\ \hline
\end{longtable} \normalsize


\subsection{Storage} \label{sec:storagecost}
\tabref{tab:StorageCost} gives the price of storage using all  types that we need.
This would be needed regardless of the compute chosen.
\tiny \begin{longtable} { |p{0.22\textwidth}  |r  |r  |r  |r  |r |} 
\caption{Total storage cost estimate \label{tab:StorageCost}}\\ 
\hline 
\textbf{Year}&\textbf{2020}&\textbf{2021}&\textbf{2022}&\textbf{2023} \\ \hline
{Fast Storage}&{\$4,736.84}&{\$4,736.84}&{\$10,428.00}&{\$62,568.00} \\ \hline
{Fast Storage Chile}&{\$0.00}&{\$0.00}&{\$0.00}&{\$62,568.00} \\ \hline
{Normal Storage}&{\$90,405.06}&{\$9,063.48}&{\$730,646.97}&{\$3,955,645.91} \\ \hline
{Latent Storage}&{\$42,363.20}&{\$74,135.60}&{\$543,948.46}&{\$3,177,069.33} \\ \hline
{Latent Storage Chile}&{\$0.00}&{\$0.00}&{\$0.00}&{\$3,837,516.59} \\ \hline
{High Latency Storage}&{\$45,549.15}&{\$50,354.27}&{\$165,965.39}&{\$727,912.22} \\ \hline
\textbf{Total}&\textbf{\$183,054.25}&\textbf{\$138,290.19}&\textbf{\$1,450,988.82}&\textbf{\$11,823,280.05} \\ \hline
\end{longtable} \normalsize


\subsection{Hosting costs} \label{sec:overheads}
\tabref{tab:overheadCost} gives the annual cost of hosting compute. This includes purchasing racks to house
new nodes etc.
\tiny \begin{longtable} { |p{0.22\textwidth}  |r  |r  |r  |r  |r |} 
\caption{Overheads(NCSA) per year based on number of cores in \tabref{tab:Inputs} and costs in \tabref{tab:overheads} assuming Xeon density from \tabref{tab:Machines}.  \label{tab:overheadCost}}\\ 
\hline 
\textbf{Year}&\textbf{2020}&\textbf{2021}&\textbf{2022}&\textbf{2023} \\ \hline
\textbf{Total Incremental cores (USA)}&\textbf{1,836}&\textbf{0}&\textbf{4,026}&\textbf{7,521} \\ \hline
\textbf{Total owned cores (USA)}&\textbf{3,528}&\textbf{3,528}&\textbf{7,554}&\textbf{15,075} \\ \hline
\textbf{Total owned nodes}&\textbf{111}&\textbf{111}&\textbf{251}&\textbf{567} \\ \hline
{Cost for hosting nodes}&{\$62,802}&{\$62,802}&{\$142,012}&{\$320,801} \\ \hline
\textbf{Total new nodes}&\textbf{58}&\textbf{0}&\textbf{140}&\textbf{317} \\ \hline
\textbf{Total new racks}&\textbf{2}&\textbf{0}&\textbf{4}&\textbf{9} \\ \hline
{Rack install cost }&{\$48,000.00}&{\$0.00}&{\$96,000.00}&{\$216,000.00} \\ \hline
\textbf{Total Overhead (NCSA)}&\textbf{\$110,802.27}&\textbf{\$62,802.27}&\textbf{\$238,012.35}&\textbf{\$536,800.80} \\ \hline
\end{longtable} \normalsize


\section{ Models}\label{sec:model}
\subsection{Sizing model}\label{sec:sizemodel}

An exhaustive and detailed mode is provided in  \citedsp{LDM-138,LDM-144} - here we concentrate on the needs for the
final years of construction. We explore the compute and storage needed to get us through commissioning and suggest a 2023 purchase for DR1,2 processing which could be pushed to operations.

\tabref{tab:Inputs} gives the annual requirements for the next few years.

\tiny \begin{longtable} { |p{0.22\textwidth}  |r  |r  |r  |r  |r  |r |} 
\caption{Various inputs for deriving costs - 2019 represents currentl holdings. \label{tab:Inputs}}\\ 
\hline 
\textbf{Year}&\textbf{2019}&\textbf{2020}&\textbf{2021}&\textbf{2022}&\textbf{2023} \\ \hline
{Core-hours Needed Total (DRP)}&{}&{4.41E+06}&{4.41E+06}&{1.12E+07}&{4.53E+07} \\ \hline
{Annual Increase}&{}&{4.41E+06}&{0.00E+00}&{6.81E+06}&{3.40E+07} \\ \hline
{Time to Process days}&{}&{100.0}&{100.0}&{100.0}&{200} \\ \hline
{Time to Process hours}&{}&{2,400}&{2,400}&{2,400}&{4,800} \\ \hline
{Instantaneous cores (DRP) Annual increase}&{1152}&{1,836}&{0}&{2,837}&{7,093} \\ \hline
{Instantaneous cores (Alerts)}&{}&{0}&{0}&{594}&{594} \\ \hline
{Cores (Alerts) Annual increase}&{}&{0}&{0}&{594}&{0} \\ \hline
{Instantaneous cores (US DAC/ Staff)}&{538}&{538}&{538}&{141}&{568} \\ \hline
{Cores (US DAC/ Staff) Annual increase}&{}&{0}&{0}&{0}&{428} \\ \hline
{Instantaneous cores (Chilean DAC)}&{}&{0}&{0}&{26}&{103} \\ \hline
{Cores (Chilean DAC) Annual increase}&{}&{0}&{0}&{26}&{78} \\ \hline
{Qserv nodes (US DAC/ Staff)}&{}&{}&{}&{14}&{95} \\ \hline
{Qserv nodes (US DAC/ Staff) Annual Increase}&{}&{}&{}&{14}&{81} \\ \hline
{Qserv nodes (Chilean DAC)}&{}&{}&{}&{14}&{95} \\ \hline
{Qserv nodes (Chilean DAC) Annual Increase}&{}&{}&{}&{14}&{81} \\ \hline
\textbf{Total Annual Increase}&\textbf{}&\textbf{1,836}&\textbf{0}&\textbf{3,457}&\textbf{7,599} \\ \hline
{Fast Storage (TB)}&{}&{12}&{24}&{50}&{206} \\ \hline
{Annual Increase (Fast)}&{}&{12}&{12}&{26}&{156} \\ \hline
{Normal Storage (TB)}&{3000}&{3391}&{3459}&{9317}&{41786} \\ \hline
{Annual Increase (Normal)}&{}&{391}&{68}&{5857}&{32469} \\ \hline
{Latent Storage  (TB)}&{}&{319}&{876}&{4057}&{20217} \\ \hline
{Annual Increase (Latent)}&{}&{319}&{557}&{3181}&{16160} \\ \hline
{High Latency (TB)}&{}&{3710}&{7727}&{18333}&{64935} \\ \hline
{Annual Increase (High Latency)}&{}&{3710}&{4017}&{10607}&{46602} \\ \hline
{Chilean DAC Fast Storage (TB)}&{}&{}&{}&{}&{156} \\ \hline
{Annual Increase (Fast Chilean DAC)}&{}&{}&{}&{}&{156} \\ \hline
{Chilean DAC Latent Storage (TB)}&{}&{}&{}&{}&{20217} \\ \hline
{Annual Increase (Latent Chilean DAC)}&{}&{}&{}&{}&{20217} \\ \hline
{Annual price decrease CPU}&{}&{10\%}&{}&{}&{} \\ \hline
{Annual price decrease Storage}&{}&{5\%}&{}&{}&{} \\ \hline
{Annual price decrease Qserv}&{}&{8\%}&{}&{}&{} \\ \hline
\end{longtable} \normalsize


\subsection{Compute and storage }\label{sec:csmodel}
We which to base our budget on reasonable well know machines for which we have well know prices.
\tabref{tab:Machines} gives an outline of a few standard machines we use and a price. This table also gives a FLOP estimate
for those machines.
\tabref{tab:Storage} gives costs for different types of storage - we will require various latency for different tasks
and those have varying costs.
These tables are used as look ups for the cost models in \secref{sec:cost}

\tiny \begin{longtable} { |p{0.22\textwidth}  |r  |r |} 
\caption{Storage types and costs used as inputs used for calculations \label{tab:Storage}}\\ 
\hline 
{Storage type }&{cost} \\ \hline
{fast -- NVME (50GB/ s each) Cost for TB  }&{\$1,000.00} \\ \hline
{normal - SATA GPFS file systems/ TB  }&{\$135.00} \\ \hline
{latency -- slower but on disk }&{\$100.00} \\ \hline
{high latency -- very slow -- on tape }&{\$64.00} \\ \hline
{}&{} \\ \hline
\end{longtable} \normalsize


In \tabref{tab:Storage} we should consider for NVME for each TB with file system servers two DDN NVME box with GPFS servers.
The price is based on the TOP performer with best price .
The Normal price is for each TB with file system disks and servers locally attached to production resources.

In the latency and high latency prices are only at NCSA: for each TB with file systems and all people/services.
The complete service not usually attached.   S3 bucket type.
Can be mounted if needed but not for production worthy speeds.
The complete service with data flowing to tape using policies.

\tiny \begin{longtable} { |p{0.22\textwidth}  |r  |r  |r  |r  |r  |r |} 
\caption{Machine types and costs used as inputs used for calculations \label{tab:Machines}}\\ 
\hline 
{Type of machine }&{Cores}&{Memory(GB)}&{GFLOP/ s}&{Cost}&{purpose/ use } \\ \hline
{xeon }&{32}&{192}&{12.672}&{\$10,000.00}&{current K8 node } \\ \hline
{qserv }&{12}&{128}&{4.104}&{\$20,000.00}&{current qserv node } \\ \hline
{small rome  }&{64}&{256}&{374.4}&{\$15,000.00}&{https:/ / www.microway.com/ product/ navion-1u-amd-epyc-gpu-server/ } \\ \hline
{large rome }&{128}&{512}&{748.8}&{\$25,000.00}& \\ \hline
{}&{}&{}&{}&{}&{} \\ \hline
{}&{}&{}&{}&{}&{} \\ \hline
\end{longtable} \normalsize


There is also an associated running cost for machines included in the total cost of ownership.
These overheads are listed in \tabref{tab:overheads}.

\tiny \begin{longtable} { |p{0.22\textwidth}  |r  |r |} 
\caption{Overhead costs per rack \label{tab:overheads}}\\ 
\hline 
\textbf{Item}&\textbf{Number/ Cost} \\ \hline
{Compute nodes in a rack }&{36} \\ \hline
{Rack initial cost has power, networking switches, networking cables, ready for machine installation-- switches last 5 years.  Will need to refresh, but rack should last entire project.  }&{\$24,000.00} \\ \hline
{ ** need to add annually: floor space for rack for 1 years.   need to renew after new nodes are racked/ stacked }&{\$300} \\ \hline
{** Need to add annually: power for 1 node for 1 yr - kw * rate * hours/ year * }&{\$348} \\ \hline
{** need to add annually: cooling for 1 node for 5 years  kw* chillded water per MTBU* hours/ year *  1KW in (MTBU) }&{\$210} \\ \hline
{** Need to add annually: maintenance for node s -- can't purchase more than what the contract has in time left.  could be included in the price of the machine, and might not be added in here.  }&{\$1,500} \\ \hline
{Cost for each machine for 1 year in a rack.   }&{\$566} \\ \hline
{**** need to add in at an annual basis.  software maintenance (oracle and other software not associated with specific node annually)  Oracle license, VM licensing.  }&{\$35,000} \\ \hline
{Compute node lifetime (years)}&{3} \\ \hline
{Storage lifetime (years)}&{5} \\ \hline
\end{longtable} \normalsize



