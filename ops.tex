\section{Operations budget estimate}\label{sec:opscost}
Based on the needs in \tabref{tab:opsInputs} and the costs in \tabref{tab:opsStorageCost} and \tabref {tab:Machines}
we get the estimate presented in \tabref{tab:opsSummary}

\tiny \begin{longtable} { |p{0.22\textwidth}  |r  |r  |r  |r  |r  |r  |r  |r  |r  |r  |r |} 
\caption{This table pulls together all the information in a high level summary for operations  for the Chile and USDF. Price factors, defined in \tabref{tab:Inputs} are applied in all cases - other input values come from \tabref{tab:opsInputs}, \tabref{tab:opsStorageCost}.
 \label{tab:opsSummary}}\\ 
\hline 
\textbf{Year  (all prices Million\$)}&\textbf{2023}&\textbf{2024}&\textbf{2025}&\textbf{2026}&\textbf{2027}&\textbf{2028}&\textbf{2029}&\textbf{2030}&\textbf{2031}&\textbf{2032} \\ \hline
{Compute (2019 pricing)}&{\$0.65}&{\$0.69}&{\$1.14}&{\$1.43}&{\$1.47}&{\$1.66}&{\$1.55}&{\$1.55}&{\$1.66}&{\$1.55} \\ \hline
{Qserv (2019 pricing)}&{\$3.52}&{\$4.84}&{\$3.72}&{\$5.08}&{\$5.48}&{\$7.20}&{\$4.48}&{\$4.36}&{\$5.56}&{\$6.52} \\ \hline
{Storage (2019 pricing)}&{\$11.82}&{\$14.49}&{\$16.68}&{\$15.69}&{\$17.36}&{\$27.62}&{\$30.34}&{\$32.53}&{\$31.55}&{\$31.96} \\ \hline
\textbf{Total (2019 pricing)}&\textbf{\$15.99}&\textbf{\$20.02}&\textbf{\$21.54}&\textbf{\$22.20}&\textbf{\$24.31}&\textbf{\$36.48}&\textbf{\$36.37}&\textbf{\$38.44}&\textbf{\$38.77}&\textbf{\$40.03} \\ \hline
{Applying price factor (CPU)}&{\$0.43}&{\$0.41}&{\$0.61}&{\$0.68}&{\$0.63}&{\$0.64}&{\$0.54}&{\$0.49}&{\$0.47}&{\$0.39} \\ \hline
{IN2P3 (50\% of US compute)}&{-\$0.21}&{-\$0.20}&{-\$0.30}&{-\$0.34}&{-\$0.32}&{-\$0.32}&{-\$0.27}&{-\$0.24}&{-\$0.23}&{-\$0.20} \\ \hline
{UKDF (25\% of compute)}&{-\$0.11}&{-\$0.10}&{-\$0.15}&{-\$0.17}&{-\$0.16}&{-\$0.16}&{-\$0.14}&{-\$0.12}&{-\$0.12}&{-\$0.10} \\ \hline
{Qserv (applying factor)}&{\$2.44}&{\$3.06}&{\$2.15}&{\$2.68}&{\$2.63}&{\$3.16}&{\$1.79}&{\$1.59}&{\$1.85}&{\$1.98} \\ \hline
{Applying price factor (Storage)}&{\$8.66}&{\$9.82}&{\$10.45}&{\$9.09}&{\$9.31}&{\$13.69}&{\$13.91}&{\$13.80}&{\$12.38}&{\$11.60} \\ \hline
{Hosting Overhead SLAC}&{\$0.45}&{\$0.51}&{\$0.51}&{\$0.49}&{\$0.51}&{\$0.66}&{\$0.63}&{\$0.62}&{\$0.58}&{\$0.56} \\ \hline
\textbf{Total budget (using price factors)}&\textbf{\$11.65}&\textbf{\$13.49}&\textbf{\$13.26}&\textbf{\$12.42}&\textbf{\$12.61}&\textbf{\$17.67}&\textbf{\$16.47}&\textbf{\$16.13}&\textbf{\$14.93}&\textbf{\$14.24} \\ \hline
\textbf{Total Operations hardware to 2032 }&\textbf{\$142.88}&\textbf{million}&&&&&&&& \\ \hline
\end{longtable} \normalsize


Again in \tabref{tab:opsSummary} we assume IN2P3 do 50\% of processing.
We have applied a compounded modest cost reduction assuming that processors  and disks get a little cheaper - that
percentage is given in \tabref{tab:Inputs}.

It must be noted that the price of disk and tape have a profound effect over 10 years. We have been fairly conservative on the base prices
in \tabref{tab:Storage}. An even bigger effect is in the compounding of the presumed fall in storage cost. Here we have used an extremely
conservative 5\% per year (\tabref{tab:Inputs}) - changing this to 15\% halves the cumulative ops estimate, setting it to 10\% brings the
total down by about 30\%.

More details on the inputs are in \secref{sec:opsdetails}


