\section{Operations budget estimate}\label{sec:opscost}
Based on the needs in \tabref{tab:opsInputs} and the costs in \tabref{tab:opsStorageCost} and \tabref {tab:Machines}
we get the estimate presented in \tabref{tab:opsSummary}.
In \tabref{tab:opsSummary} we should note that IN2P3 do 50\%  and UKDF do 25\% of the processing so we reduce the processing cost by three quarters.
This does not reduce the storage cost.

\tiny \begin{longtable} { |p{0.22\textwidth}  |r  |r  |r  |r  |r  |r  |r  |r  |r  |r  |r |} 
\caption{This table pulls together all the information in a high level summary for operations - in this table Xeon pricing(see \tabref{opsXeon}) is used since that is the more expensive but better known option. Price factors, defined in \tabref{tab:Inputs} are applied in all cases - other input values come from \tabref{tab:opsInputs}, \tabref{tab:opsStorageCost}.
 \label{tab:opsSummary}}\\ 
\hline 
\textbf{Year}&\textbf{2023}&\textbf{2024}&\textbf{2025}&\textbf{2026}&\textbf{2027}&\textbf{2028}&\textbf{2029}&\textbf{2030}&\textbf{2031}&\textbf{2032} \\ \hline
{Compute (2019 pricing)}&{\$2,830,000}&{\$3,000,000}&{\$4,720,000}&{\$6,210,000}&{\$6,370,000}&{\$6,980,000}&{\$6,750,000}&{\$6,750,000}&{\$6,980,000}&{\$6,750,000} \\ \hline
{Applying price factor (CPU)}&{\$1,981,000}&{\$1,968,300}&{\$2,787,113}&{\$3,300,249}&{\$3,046,751}&{\$3,004,661}&{\$2,615,088}&{\$2,353,579}&{\$2,190,398}&{\$1,906,399} \\ \hline
{IN2P3 (50\% of compute)}&{-\$990,500}&{-\$984,150}&{-\$1,393,556}&{-\$1,650,124}&{-\$1,523,376}&{-\$1,502,331}&{-\$1,307,544}&{-\$1,176,790}&{-\$1,095,199}&{-\$953,200} \\ \hline
{Qserv (2019 pricing)}&{\$3,240,000}&{\$4,840,000}&{\$3,720,000}&{\$4,800,000}&{\$5,480,000}&{\$7,200,000}&{\$4,200,000}&{\$4,360,000}&{\$5,560,000}&{\$6,240,000} \\ \hline
{Qserv (applying factor)}&{\$2,511,000}&{\$3,543,336}&{\$2,519,136}&{\$3,006,711}&{\$3,175,212}&{\$3,858,925}&{\$2,082,212}&{\$1,999,419}&{\$2,358,489}&{\$2,448,417} \\ \hline
{Storage (2019 pricing)}&{\$11,316,420}&&&&&&&&& \\ \hline
{Applying price factor (Storage)}&{\$9,618,957}&{\$0}&{\$0}&{\$0}&{\$0}&{\$0}&{\$0}&{\$0}&{\$0}&{\$0} \\ \hline
{Hosting Overhead NCSA
}&{\$537,367}&{\$0}&&&&&&&& \\ \hline
\textbf{Total budget (using price factors)}&\textbf{\$13,657,824}&\textbf{\$4,527,486}&\textbf{\$3,912,692}&\textbf{\$4,656,835}&\textbf{\$4,698,587}&\textbf{\$5,361,256}&\textbf{\$3,389,756}&\textbf{\$3,176,209}&\textbf{\$3,453,688}&\textbf{\$3,401,617} \\ \hline
\textbf{Total Operations hardware to 2032 }&\textbf{\$50,235,949}&&&&&&&&& \\ \hline
\end{longtable} \normalsize


Again in \tabref{tab:opsSummary} we assume IN2P3 do 50\% of processing (see \tabref{tab:opsSumUSDF}).
We have applied a compounded modest cost reduction assuming that processors  and disks get a little cheaper - that
percentage is given in \tabref{tab:Inputs}.

It must be noted that the price of disk and tape have a profound effect over 10 years. We have been fairly conservative on the base prices
in \tabref{tab:Storage}. An even bigger effect is in the compounding of the presumed fall in storage cost. Here we have used an extremely
conservative 5\% per year (\tabref{tab:Inputs}) - changing this to 15\% halves the cumulative ops estimate, setting it to 10\% brings the
total down by about 30\%.

\subsection{Cloud costs}
In addition there are some cloud costs. We run certain jobs and host websites on Amazon and Google. In operations
the validation team may also wish to run simulations on cloud resources. This estimate is in \tabref{tab:cloud}.

\tiny \begin{longtable} { |p{0.22\textwidth}  |r  |r  |r  |r  |r  |r  |r  |r  |r  |r  |r  |r |} 
\caption{We have on going cloud costs and assume some other activities may be on cloud in the future - we make an estimate of those costs here. \label{tab:cloud}}\\ 
\hline 
\textbf{Year  (all prices Million\$)}&\textbf{2024}&\textbf{2025}&\textbf{2026}&\textbf{2027}&\textbf{2028}&\textbf{2029}&\textbf{2030}&\textbf{2031}&\textbf{2032}&\textbf{2033}&\textbf{2034} \\ \hline
{Jira Cloud}&{\$80,000}&{\$80,000}&{\$80,000}&{\$80,000}&{\$80,000}&{\$80,000}&{\$80,000}&{\$80,000}&{\$80,000}&{\$80,000}&{\$80,000} \\ \hline
{Current actuals}&{\$20,000}&{\$20,000}&{\$20,000}&{\$20,000}&{\$20,000}&{\$20,000}&{\$20,000}&{\$20,000}&{\$20,000}&{\$20,000}&{\$20,000} \\ \hline
{RPF sims, V&V}&{\$10,000}&{\$10,000}&{\$10,000}&{\$10,000}&{\$10,000}&{\$10,000}&{\$10,000}&{\$10,000}&{\$10,000}&{\$10,000}& \\ \hline
\textbf{Total}&\textbf{\$110,000}&\textbf{\$110,000}&\textbf{\$110,000}&\textbf{\$110,000}&\textbf{\$110,000}&\textbf{\$110,000}&\textbf{\$110,000}&\textbf{\$110,000}&\textbf{\$110,000}&\textbf{\$110,000}&\textbf{\$100,000} \\ \hline
{As M\$}&{\$0.11}&{\$0.11}&{\$0.11}&{\$0.11}&{\$0.11}&{\$0.11}&{\$0.11}&{\$0.11}&{\$0.11}&{\$0.11}&{\$0.10} \\ \hline
{}&{Sims }&{8 nodes 12 CPUs 5 days}&&&&&&&&& \\ \hline
\end{longtable} \normalsize



More details on the inputs are in \secref{sec:opsdetails}

\subsection{US and Chile}
While \tabref{tab:opsSummary} present the total ops cost for Rubin Observatory a fraction of this is in Chile and would potentially remain an NSF cost in operations. \tabref{tab:opsSumUSDF} presents just the US Data Facility budget  and
\tabref{tab:opsChileR} presents the Chile budget.

\tiny \begin{longtable} { |p{0.22\textwidth}  |r  |r  |r  |r  |r  |r  |r  |r  |r  |r  |r |} 
\caption{This table pulls together all the information in a high level summary for USDF operations - in this table Rome pricing(see \tabref{tab:opsRomeUSDF}). Price factors, defined in \tabref{tab:Inputs} are applied in all cases - other input values come from \tabref{tab:opsInputs}, \tabref{tab:opsStorageUSDF}.
 \label{tab:opsSumUSDF}}\\ 
\hline 
\textbf{Year  (all prices Million\$)}&\textbf{2024}&\textbf{2025}&\textbf{2026}&\textbf{2027}&\textbf{2028}&\textbf{2029}&\textbf{2030}&\textbf{2031}&\textbf{2032}&\textbf{2033} \\ \hline
{Compute (2019 pricing)}&{\$0.65}&{\$0.69}&{\$1.14}&{\$1.43}&{\$1.47}&{\$1.66}&{\$1.55}&{\$1.55}&{\$1.66}&{\$1.55} \\ \hline
{Qserv (2019 pricing)}&{\$1.62}&{\$2.42}&{\$1.86}&{\$2.40}&{\$2.74}&{\$3.60}&{\$2.10}&{\$2.18}&{\$2.78}&{\$3.12} \\ \hline
{Storage (2019 pricing)}&{\$7.92}&{\$9.73}&{\$11.22}&{\$10.94}&{\$12.62}&{\$18.97}&{\$20.83}&{\$22.33}&{\$22.05}&{\$22.47} \\ \hline
\textbf{Total (2019 pricing)}&\textbf{\$10.19}&\textbf{\$12.84}&\textbf{\$14.22}&\textbf{\$14.77}&\textbf{\$16.83}&\textbf{\$24.23}&\textbf{\$24.48}&\textbf{\$26.06}&\textbf{\$26.49}&\textbf{\$27.14} \\ \hline
{Applying price factor (CPU)}&{\$0.43}&{\$0.41}&{\$0.61}&{\$0.68}&{\$0.63}&{\$0.64}&{\$0.54}&{\$0.49}&{\$0.47}&{\$0.39} \\ \hline
{IN2P3 (50\% of compute)}&{-\$0.21}&{-\$0.20}&{-\$0.30}&{-\$0.34}&{-\$0.32}&{-\$0.32}&{-\$0.27}&{-\$0.24}&{-\$0.23}&{-\$0.20} \\ \hline
{UKDF (25\% of compute)}&{-\$0.11}&{-\$0.10}&{-\$0.15}&{-\$0.17}&{-\$0.16}&{-\$0.16}&{-\$0.14}&{-\$0.12}&{-\$0.12}&{-\$0.10} \\ \hline
{Qserv (applying factor)}&{\$1.12}&{\$1.53}&{\$1.07}&{\$1.26}&{\$1.32}&{\$1.58}&{\$0.84}&{\$0.80}&{\$0.93}&{\$0.95} \\ \hline
{Applying price factor (Storage)}&{\$5.80}&{\$6.59}&{\$7.03}&{\$6.34}&{\$6.76}&{\$9.41}&{\$9.55}&{\$9.47}&{\$8.65}&{\$8.16} \\ \hline
{Hosting Overhead SLAC
}&{0.5}&{0.5}&{0.5}&{0.5}&{0.5}&{0.7}&{0.6}&{0.6}&{0.6}&{0.6} \\ \hline
\textbf{Total budget (using price factors)}&\textbf{\$7.48}&\textbf{\$8.74}&\textbf{\$8.77}&\textbf{\$8.26}&\textbf{\$8.75}&\textbf{\$11.80}&\textbf{\$11.16}&\textbf{\$11.01}&\textbf{\$10.28}&\textbf{\$9.76} \\ \hline
\textbf{Total Operations hardware to 2032 }&\textbf{\$96.01}&\textbf{million}&&&&&&&& \\ \hline
\end{longtable} \normalsize


Note that in \tabref{tab:opsChileR} a fraction of the USDF storage is considered, big enough for Raws and a cache of products. It also does not include a Qserv. See also \tabref{tab:opsSumChile}
\tiny \begin{longtable} { |p{0.22\textwidth}  |r  |r  |r  |r  |r  |r  |r  |r  |r  |r  |r |} 
\caption{"This table pulls together all the information in a high level summary 
for Chile operations - in this table Rome pricing(see 
\tabref{tab:opsRomeChile}) is used  Price factors, defined in 
\tabref{tab:Inputs} are applied in all cases - other input values come 
from \tabref{tab:opsInputs}, \tabref{tab:opsStorageChile}. \label{tab:opsChileR}}\\ 
\hline 
\textbf{Year  (all prices Million\$)}&\textbf{2024}&\textbf{2025}&\textbf{2026}&\textbf{2027}&\textbf{2028}&\textbf{2029}&\textbf{2030}&\textbf{2031}&\textbf{2032}&\textbf{2033} \\ \hline
{Compute (2020 pricing)}&{\$0.04}&{\$0.04}&{\$0.04}&{\$0.08}&{\$0.08}&{\$0.08}&{\$0.08}&{\$0.08}&{\$0.08}&{\$0.08} \\ \hline
{Storage (2020 pricing)}&{\$2.32}&{\$1.85}&{\$1.73}&{\$0.77}&{\$0.76}&{\$3.07}&{\$2.60}&{\$2.48}&{\$1.52}&{\$1.51} \\ \hline
\textbf{Total (2020 pricing)}&\textbf{\$2.35}&\textbf{\$1.88}&\textbf{\$1.77}&\textbf{\$0.85}&\textbf{\$0.83}&\textbf{\$3.15}&\textbf{\$2.68}&\textbf{\$2.56}&\textbf{\$1.60}&\textbf{\$1.59} \\ \hline
{Applying price factor (CPU)}&{\$0.03}&{\$0.02}&{\$0.02}&{\$0.04}&{\$0.03}&{\$0.03}&{\$0.02}&{\$0.02}&{\$0.02}&{\$0.02} \\ \hline
{Applying price factor (Storage)}&{\$1.69}&{\$1.25}&{\$1.08}&{\$0.45}&{\$0.41}&{\$1.52}&{\$1.19}&{\$1.05}&{\$0.60}&{\$0.55} \\ \hline
{Overhead hosting Chile}&{\$0.30}&{\$0.31}&{\$0.30}&{\$0.27}&{\$0.29}&{\$0.52}&{\$0.35}&{\$0.33}&{\$0.29}&{\$0.33} \\ \hline
\textbf{Total budget (using price factors)}&\textbf{\$2.02}&\textbf{\$1.59}&\textbf{\$1.40}&\textbf{\$0.75}&\textbf{\$0.73}&\textbf{\$2.07}&\textbf{\$1.57}&\textbf{\$1.40}&\textbf{\$0.90}&\textbf{\$0.90} \\ \hline
\textbf{Total Chile hardware to 2033}&\textbf{\$13.33}&\textbf{million}&\textbf{to 2035}&\textbf{\$14.11}&&&&&& \\ \hline
\end{longtable} \normalsize

